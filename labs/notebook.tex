
% Default to the notebook output style

    


% Inherit from the specified cell style.




    
\documentclass[11pt]{article}

    
    
    \usepackage[T1]{fontenc}
    % Nicer default font (+ math font) than Computer Modern for most use cases
    \usepackage{mathpazo}

    % Basic figure setup, for now with no caption control since it's done
    % automatically by Pandoc (which extracts ![](path) syntax from Markdown).
    \usepackage{graphicx}
    % We will generate all images so they have a width \maxwidth. This means
    % that they will get their normal width if they fit onto the page, but
    % are scaled down if they would overflow the margins.
    \makeatletter
    \def\maxwidth{\ifdim\Gin@nat@width>\linewidth\linewidth
    \else\Gin@nat@width\fi}
    \makeatother
    \let\Oldincludegraphics\includegraphics
    % Set max figure width to be 80% of text width, for now hardcoded.
    \renewcommand{\includegraphics}[1]{\Oldincludegraphics[width=.8\maxwidth]{#1}}
    % Ensure that by default, figures have no caption (until we provide a
    % proper Figure object with a Caption API and a way to capture that
    % in the conversion process - todo).
    \usepackage{caption}
    \DeclareCaptionLabelFormat{nolabel}{}
    \captionsetup{labelformat=nolabel}

    \usepackage{adjustbox} % Used to constrain images to a maximum size 
    \usepackage{xcolor} % Allow colors to be defined
    \usepackage{enumerate} % Needed for markdown enumerations to work
    \usepackage{geometry} % Used to adjust the document margins
    \usepackage{amsmath} % Equations
    \usepackage{amssymb} % Equations
    \usepackage{textcomp} % defines textquotesingle
    % Hack from http://tex.stackexchange.com/a/47451/13684:
    \AtBeginDocument{%
        \def\PYZsq{\textquotesingle}% Upright quotes in Pygmentized code
    }
    \usepackage{upquote} % Upright quotes for verbatim code
    \usepackage{eurosym} % defines \euro
    \usepackage[mathletters]{ucs} % Extended unicode (utf-8) support
    \usepackage[utf8x]{inputenc} % Allow utf-8 characters in the tex document
    \usepackage{fancyvrb} % verbatim replacement that allows latex
    \usepackage{grffile} % extends the file name processing of package graphics 
                         % to support a larger range 
    % The hyperref package gives us a pdf with properly built
    % internal navigation ('pdf bookmarks' for the table of contents,
    % internal cross-reference links, web links for URLs, etc.)
    \usepackage{hyperref}
    \usepackage{longtable} % longtable support required by pandoc >1.10
    \usepackage{booktabs}  % table support for pandoc > 1.12.2
    \usepackage[inline]{enumitem} % IRkernel/repr support (it uses the enumerate* environment)
    \usepackage[normalem]{ulem} % ulem is needed to support strikethroughs (\sout)
                                % normalem makes italics be italics, not underlines
    

    
    
    % Colors for the hyperref package
    \definecolor{urlcolor}{rgb}{0,.145,.698}
    \definecolor{linkcolor}{rgb}{.71,0.21,0.01}
    \definecolor{citecolor}{rgb}{.12,.54,.11}

    % ANSI colors
    \definecolor{ansi-black}{HTML}{3E424D}
    \definecolor{ansi-black-intense}{HTML}{282C36}
    \definecolor{ansi-red}{HTML}{E75C58}
    \definecolor{ansi-red-intense}{HTML}{B22B31}
    \definecolor{ansi-green}{HTML}{00A250}
    \definecolor{ansi-green-intense}{HTML}{007427}
    \definecolor{ansi-yellow}{HTML}{DDB62B}
    \definecolor{ansi-yellow-intense}{HTML}{B27D12}
    \definecolor{ansi-blue}{HTML}{208FFB}
    \definecolor{ansi-blue-intense}{HTML}{0065CA}
    \definecolor{ansi-magenta}{HTML}{D160C4}
    \definecolor{ansi-magenta-intense}{HTML}{A03196}
    \definecolor{ansi-cyan}{HTML}{60C6C8}
    \definecolor{ansi-cyan-intense}{HTML}{258F8F}
    \definecolor{ansi-white}{HTML}{C5C1B4}
    \definecolor{ansi-white-intense}{HTML}{A1A6B2}

    % commands and environments needed by pandoc snippets
    % extracted from the output of `pandoc -s`
    \providecommand{\tightlist}{%
      \setlength{\itemsep}{0pt}\setlength{\parskip}{0pt}}
    \DefineVerbatimEnvironment{Highlighting}{Verbatim}{commandchars=\\\{\}}
    % Add ',fontsize=\small' for more characters per line
    \newenvironment{Shaded}{}{}
    \newcommand{\KeywordTok}[1]{\textcolor[rgb]{0.00,0.44,0.13}{\textbf{{#1}}}}
    \newcommand{\DataTypeTok}[1]{\textcolor[rgb]{0.56,0.13,0.00}{{#1}}}
    \newcommand{\DecValTok}[1]{\textcolor[rgb]{0.25,0.63,0.44}{{#1}}}
    \newcommand{\BaseNTok}[1]{\textcolor[rgb]{0.25,0.63,0.44}{{#1}}}
    \newcommand{\FloatTok}[1]{\textcolor[rgb]{0.25,0.63,0.44}{{#1}}}
    \newcommand{\CharTok}[1]{\textcolor[rgb]{0.25,0.44,0.63}{{#1}}}
    \newcommand{\StringTok}[1]{\textcolor[rgb]{0.25,0.44,0.63}{{#1}}}
    \newcommand{\CommentTok}[1]{\textcolor[rgb]{0.38,0.63,0.69}{\textit{{#1}}}}
    \newcommand{\OtherTok}[1]{\textcolor[rgb]{0.00,0.44,0.13}{{#1}}}
    \newcommand{\AlertTok}[1]{\textcolor[rgb]{1.00,0.00,0.00}{\textbf{{#1}}}}
    \newcommand{\FunctionTok}[1]{\textcolor[rgb]{0.02,0.16,0.49}{{#1}}}
    \newcommand{\RegionMarkerTok}[1]{{#1}}
    \newcommand{\ErrorTok}[1]{\textcolor[rgb]{1.00,0.00,0.00}{\textbf{{#1}}}}
    \newcommand{\NormalTok}[1]{{#1}}
    
    % Additional commands for more recent versions of Pandoc
    \newcommand{\ConstantTok}[1]{\textcolor[rgb]{0.53,0.00,0.00}{{#1}}}
    \newcommand{\SpecialCharTok}[1]{\textcolor[rgb]{0.25,0.44,0.63}{{#1}}}
    \newcommand{\VerbatimStringTok}[1]{\textcolor[rgb]{0.25,0.44,0.63}{{#1}}}
    \newcommand{\SpecialStringTok}[1]{\textcolor[rgb]{0.73,0.40,0.53}{{#1}}}
    \newcommand{\ImportTok}[1]{{#1}}
    \newcommand{\DocumentationTok}[1]{\textcolor[rgb]{0.73,0.13,0.13}{\textit{{#1}}}}
    \newcommand{\AnnotationTok}[1]{\textcolor[rgb]{0.38,0.63,0.69}{\textbf{\textit{{#1}}}}}
    \newcommand{\CommentVarTok}[1]{\textcolor[rgb]{0.38,0.63,0.69}{\textbf{\textit{{#1}}}}}
    \newcommand{\VariableTok}[1]{\textcolor[rgb]{0.10,0.09,0.49}{{#1}}}
    \newcommand{\ControlFlowTok}[1]{\textcolor[rgb]{0.00,0.44,0.13}{\textbf{{#1}}}}
    \newcommand{\OperatorTok}[1]{\textcolor[rgb]{0.40,0.40,0.40}{{#1}}}
    \newcommand{\BuiltInTok}[1]{{#1}}
    \newcommand{\ExtensionTok}[1]{{#1}}
    \newcommand{\PreprocessorTok}[1]{\textcolor[rgb]{0.74,0.48,0.00}{{#1}}}
    \newcommand{\AttributeTok}[1]{\textcolor[rgb]{0.49,0.56,0.16}{{#1}}}
    \newcommand{\InformationTok}[1]{\textcolor[rgb]{0.38,0.63,0.69}{\textbf{\textit{{#1}}}}}
    \newcommand{\WarningTok}[1]{\textcolor[rgb]{0.38,0.63,0.69}{\textbf{\textit{{#1}}}}}
    
    
    % Define a nice break command that doesn't care if a line doesn't already
    % exist.
    \def\br{\hspace*{\fill} \\* }
    % Math Jax compatability definitions
    \def\gt{>}
    \def\lt{<}
    % Document parameters
    \title{Websearch\_2018\_lab2-solution}
    
    
    

    % Pygments definitions
    
\makeatletter
\def\PY@reset{\let\PY@it=\relax \let\PY@bf=\relax%
    \let\PY@ul=\relax \let\PY@tc=\relax%
    \let\PY@bc=\relax \let\PY@ff=\relax}
\def\PY@tok#1{\csname PY@tok@#1\endcsname}
\def\PY@toks#1+{\ifx\relax#1\empty\else%
    \PY@tok{#1}\expandafter\PY@toks\fi}
\def\PY@do#1{\PY@bc{\PY@tc{\PY@ul{%
    \PY@it{\PY@bf{\PY@ff{#1}}}}}}}
\def\PY#1#2{\PY@reset\PY@toks#1+\relax+\PY@do{#2}}

\expandafter\def\csname PY@tok@w\endcsname{\def\PY@tc##1{\textcolor[rgb]{0.73,0.73,0.73}{##1}}}
\expandafter\def\csname PY@tok@c\endcsname{\let\PY@it=\textit\def\PY@tc##1{\textcolor[rgb]{0.25,0.50,0.50}{##1}}}
\expandafter\def\csname PY@tok@cp\endcsname{\def\PY@tc##1{\textcolor[rgb]{0.74,0.48,0.00}{##1}}}
\expandafter\def\csname PY@tok@k\endcsname{\let\PY@bf=\textbf\def\PY@tc##1{\textcolor[rgb]{0.00,0.50,0.00}{##1}}}
\expandafter\def\csname PY@tok@kp\endcsname{\def\PY@tc##1{\textcolor[rgb]{0.00,0.50,0.00}{##1}}}
\expandafter\def\csname PY@tok@kt\endcsname{\def\PY@tc##1{\textcolor[rgb]{0.69,0.00,0.25}{##1}}}
\expandafter\def\csname PY@tok@o\endcsname{\def\PY@tc##1{\textcolor[rgb]{0.40,0.40,0.40}{##1}}}
\expandafter\def\csname PY@tok@ow\endcsname{\let\PY@bf=\textbf\def\PY@tc##1{\textcolor[rgb]{0.67,0.13,1.00}{##1}}}
\expandafter\def\csname PY@tok@nb\endcsname{\def\PY@tc##1{\textcolor[rgb]{0.00,0.50,0.00}{##1}}}
\expandafter\def\csname PY@tok@nf\endcsname{\def\PY@tc##1{\textcolor[rgb]{0.00,0.00,1.00}{##1}}}
\expandafter\def\csname PY@tok@nc\endcsname{\let\PY@bf=\textbf\def\PY@tc##1{\textcolor[rgb]{0.00,0.00,1.00}{##1}}}
\expandafter\def\csname PY@tok@nn\endcsname{\let\PY@bf=\textbf\def\PY@tc##1{\textcolor[rgb]{0.00,0.00,1.00}{##1}}}
\expandafter\def\csname PY@tok@ne\endcsname{\let\PY@bf=\textbf\def\PY@tc##1{\textcolor[rgb]{0.82,0.25,0.23}{##1}}}
\expandafter\def\csname PY@tok@nv\endcsname{\def\PY@tc##1{\textcolor[rgb]{0.10,0.09,0.49}{##1}}}
\expandafter\def\csname PY@tok@no\endcsname{\def\PY@tc##1{\textcolor[rgb]{0.53,0.00,0.00}{##1}}}
\expandafter\def\csname PY@tok@nl\endcsname{\def\PY@tc##1{\textcolor[rgb]{0.63,0.63,0.00}{##1}}}
\expandafter\def\csname PY@tok@ni\endcsname{\let\PY@bf=\textbf\def\PY@tc##1{\textcolor[rgb]{0.60,0.60,0.60}{##1}}}
\expandafter\def\csname PY@tok@na\endcsname{\def\PY@tc##1{\textcolor[rgb]{0.49,0.56,0.16}{##1}}}
\expandafter\def\csname PY@tok@nt\endcsname{\let\PY@bf=\textbf\def\PY@tc##1{\textcolor[rgb]{0.00,0.50,0.00}{##1}}}
\expandafter\def\csname PY@tok@nd\endcsname{\def\PY@tc##1{\textcolor[rgb]{0.67,0.13,1.00}{##1}}}
\expandafter\def\csname PY@tok@s\endcsname{\def\PY@tc##1{\textcolor[rgb]{0.73,0.13,0.13}{##1}}}
\expandafter\def\csname PY@tok@sd\endcsname{\let\PY@it=\textit\def\PY@tc##1{\textcolor[rgb]{0.73,0.13,0.13}{##1}}}
\expandafter\def\csname PY@tok@si\endcsname{\let\PY@bf=\textbf\def\PY@tc##1{\textcolor[rgb]{0.73,0.40,0.53}{##1}}}
\expandafter\def\csname PY@tok@se\endcsname{\let\PY@bf=\textbf\def\PY@tc##1{\textcolor[rgb]{0.73,0.40,0.13}{##1}}}
\expandafter\def\csname PY@tok@sr\endcsname{\def\PY@tc##1{\textcolor[rgb]{0.73,0.40,0.53}{##1}}}
\expandafter\def\csname PY@tok@ss\endcsname{\def\PY@tc##1{\textcolor[rgb]{0.10,0.09,0.49}{##1}}}
\expandafter\def\csname PY@tok@sx\endcsname{\def\PY@tc##1{\textcolor[rgb]{0.00,0.50,0.00}{##1}}}
\expandafter\def\csname PY@tok@m\endcsname{\def\PY@tc##1{\textcolor[rgb]{0.40,0.40,0.40}{##1}}}
\expandafter\def\csname PY@tok@gh\endcsname{\let\PY@bf=\textbf\def\PY@tc##1{\textcolor[rgb]{0.00,0.00,0.50}{##1}}}
\expandafter\def\csname PY@tok@gu\endcsname{\let\PY@bf=\textbf\def\PY@tc##1{\textcolor[rgb]{0.50,0.00,0.50}{##1}}}
\expandafter\def\csname PY@tok@gd\endcsname{\def\PY@tc##1{\textcolor[rgb]{0.63,0.00,0.00}{##1}}}
\expandafter\def\csname PY@tok@gi\endcsname{\def\PY@tc##1{\textcolor[rgb]{0.00,0.63,0.00}{##1}}}
\expandafter\def\csname PY@tok@gr\endcsname{\def\PY@tc##1{\textcolor[rgb]{1.00,0.00,0.00}{##1}}}
\expandafter\def\csname PY@tok@ge\endcsname{\let\PY@it=\textit}
\expandafter\def\csname PY@tok@gs\endcsname{\let\PY@bf=\textbf}
\expandafter\def\csname PY@tok@gp\endcsname{\let\PY@bf=\textbf\def\PY@tc##1{\textcolor[rgb]{0.00,0.00,0.50}{##1}}}
\expandafter\def\csname PY@tok@go\endcsname{\def\PY@tc##1{\textcolor[rgb]{0.53,0.53,0.53}{##1}}}
\expandafter\def\csname PY@tok@gt\endcsname{\def\PY@tc##1{\textcolor[rgb]{0.00,0.27,0.87}{##1}}}
\expandafter\def\csname PY@tok@err\endcsname{\def\PY@bc##1{\setlength{\fboxsep}{0pt}\fcolorbox[rgb]{1.00,0.00,0.00}{1,1,1}{\strut ##1}}}
\expandafter\def\csname PY@tok@kc\endcsname{\let\PY@bf=\textbf\def\PY@tc##1{\textcolor[rgb]{0.00,0.50,0.00}{##1}}}
\expandafter\def\csname PY@tok@kd\endcsname{\let\PY@bf=\textbf\def\PY@tc##1{\textcolor[rgb]{0.00,0.50,0.00}{##1}}}
\expandafter\def\csname PY@tok@kn\endcsname{\let\PY@bf=\textbf\def\PY@tc##1{\textcolor[rgb]{0.00,0.50,0.00}{##1}}}
\expandafter\def\csname PY@tok@kr\endcsname{\let\PY@bf=\textbf\def\PY@tc##1{\textcolor[rgb]{0.00,0.50,0.00}{##1}}}
\expandafter\def\csname PY@tok@bp\endcsname{\def\PY@tc##1{\textcolor[rgb]{0.00,0.50,0.00}{##1}}}
\expandafter\def\csname PY@tok@fm\endcsname{\def\PY@tc##1{\textcolor[rgb]{0.00,0.00,1.00}{##1}}}
\expandafter\def\csname PY@tok@vc\endcsname{\def\PY@tc##1{\textcolor[rgb]{0.10,0.09,0.49}{##1}}}
\expandafter\def\csname PY@tok@vg\endcsname{\def\PY@tc##1{\textcolor[rgb]{0.10,0.09,0.49}{##1}}}
\expandafter\def\csname PY@tok@vi\endcsname{\def\PY@tc##1{\textcolor[rgb]{0.10,0.09,0.49}{##1}}}
\expandafter\def\csname PY@tok@vm\endcsname{\def\PY@tc##1{\textcolor[rgb]{0.10,0.09,0.49}{##1}}}
\expandafter\def\csname PY@tok@sa\endcsname{\def\PY@tc##1{\textcolor[rgb]{0.73,0.13,0.13}{##1}}}
\expandafter\def\csname PY@tok@sb\endcsname{\def\PY@tc##1{\textcolor[rgb]{0.73,0.13,0.13}{##1}}}
\expandafter\def\csname PY@tok@sc\endcsname{\def\PY@tc##1{\textcolor[rgb]{0.73,0.13,0.13}{##1}}}
\expandafter\def\csname PY@tok@dl\endcsname{\def\PY@tc##1{\textcolor[rgb]{0.73,0.13,0.13}{##1}}}
\expandafter\def\csname PY@tok@s2\endcsname{\def\PY@tc##1{\textcolor[rgb]{0.73,0.13,0.13}{##1}}}
\expandafter\def\csname PY@tok@sh\endcsname{\def\PY@tc##1{\textcolor[rgb]{0.73,0.13,0.13}{##1}}}
\expandafter\def\csname PY@tok@s1\endcsname{\def\PY@tc##1{\textcolor[rgb]{0.73,0.13,0.13}{##1}}}
\expandafter\def\csname PY@tok@mb\endcsname{\def\PY@tc##1{\textcolor[rgb]{0.40,0.40,0.40}{##1}}}
\expandafter\def\csname PY@tok@mf\endcsname{\def\PY@tc##1{\textcolor[rgb]{0.40,0.40,0.40}{##1}}}
\expandafter\def\csname PY@tok@mh\endcsname{\def\PY@tc##1{\textcolor[rgb]{0.40,0.40,0.40}{##1}}}
\expandafter\def\csname PY@tok@mi\endcsname{\def\PY@tc##1{\textcolor[rgb]{0.40,0.40,0.40}{##1}}}
\expandafter\def\csname PY@tok@il\endcsname{\def\PY@tc##1{\textcolor[rgb]{0.40,0.40,0.40}{##1}}}
\expandafter\def\csname PY@tok@mo\endcsname{\def\PY@tc##1{\textcolor[rgb]{0.40,0.40,0.40}{##1}}}
\expandafter\def\csname PY@tok@ch\endcsname{\let\PY@it=\textit\def\PY@tc##1{\textcolor[rgb]{0.25,0.50,0.50}{##1}}}
\expandafter\def\csname PY@tok@cm\endcsname{\let\PY@it=\textit\def\PY@tc##1{\textcolor[rgb]{0.25,0.50,0.50}{##1}}}
\expandafter\def\csname PY@tok@cpf\endcsname{\let\PY@it=\textit\def\PY@tc##1{\textcolor[rgb]{0.25,0.50,0.50}{##1}}}
\expandafter\def\csname PY@tok@c1\endcsname{\let\PY@it=\textit\def\PY@tc##1{\textcolor[rgb]{0.25,0.50,0.50}{##1}}}
\expandafter\def\csname PY@tok@cs\endcsname{\let\PY@it=\textit\def\PY@tc##1{\textcolor[rgb]{0.25,0.50,0.50}{##1}}}

\def\PYZbs{\char`\\}
\def\PYZus{\char`\_}
\def\PYZob{\char`\{}
\def\PYZcb{\char`\}}
\def\PYZca{\char`\^}
\def\PYZam{\char`\&}
\def\PYZlt{\char`\<}
\def\PYZgt{\char`\>}
\def\PYZsh{\char`\#}
\def\PYZpc{\char`\%}
\def\PYZdl{\char`\$}
\def\PYZhy{\char`\-}
\def\PYZsq{\char`\'}
\def\PYZdq{\char`\"}
\def\PYZti{\char`\~}
% for compatibility with earlier versions
\def\PYZat{@}
\def\PYZlb{[}
\def\PYZrb{]}
\makeatother


    % Exact colors from NB
    \definecolor{incolor}{rgb}{0.0, 0.0, 0.5}
    \definecolor{outcolor}{rgb}{0.545, 0.0, 0.0}



    
    % Prevent overflowing lines due to hard-to-break entities
    \sloppy 
    % Setup hyperref package
    \hypersetup{
      breaklinks=true,  % so long urls are correctly broken across lines
      colorlinks=true,
      urlcolor=urlcolor,
      linkcolor=linkcolor,
      citecolor=citecolor,
      }
    % Slightly bigger margins than the latex defaults
    
    \geometry{verbose,tmargin=1in,bmargin=1in,lmargin=1in,rmargin=1in}
    
    

    \begin{document}
    
    
    \maketitle
    
    

    
    \hypertarget{web-search-2018---tutorial-2-basic-search-spaces}{%
\section{Web Search 2018 - Tutorial 2: Basic search
spaces}\label{web-search-2018---tutorial-2-basic-search-spaces}}

\hypertarget{contents}{%
\subsection{Contents}\label{contents}}

\begin{enumerate}
\def\labelenumi{\arabic{enumi}.}
\tightlist
\item
  Section \ref{head1}
\item
  Section \ref{head2}
\item
  Section \ref{head21}
\item
  Section \ref{head22}
\item
  Section \ref{head23}
\item
  Section \ref{head3}
\item
  Section \ref{head31}
\item
  Section \ref{head32}
\item
  Section \ref{head4}
\item
  Section \ref{head41}
\item
  Section \ref{head42}
\item
  Section \ref{head43}
\item
  Section \ref{head44}
\item
  Section \ref{head5}
\end{enumerate}

\hypertarget{overview}{%
\subsection{ Overview}\label{overview}}

The goal of this lab is to develop an architecture to search Text and
Images, on different vector spaces.

We provide you with an example of how you can represent an image using
the Histogram of Oriented Gradients image descriptor. Since one image is
represented as a vector, you can represent multiple images as a matrix.
Starting from this example, you can add other descriptors such as (HoC
and GIST).

In summary, you should implement the following functionality:

\begin{itemize}
\tightlist
\item
  Individual feature vectors \textbf{normalization};
\item
  Extend the provided code to extract features from images of the
  project dataset, and \textbf{store them as a matrix};
\item
  Add support for \textbf{additional features}, apart from HoG: (HoC and
  GIST for now);
\item
  \textbf{Bag of Words} representation for words. Store texts as a
  sparse matrix;
\item
  Implement a straighforward search function for individual search
  spaces: Given an image/text as input, retrieve the \textbf{top-k
  nearest documents}, using an \textbf{adequate distance function}.
\end{itemize}

We strongly encourage you to use the
\href{http://scikit-learn.org/stable/}{scikit-learn} library. Most of
the functionalities that you will need for you project are already
implemented there.

    \begin{Verbatim}[commandchars=\\\{\}]
{\color{incolor}In [{\color{incolor}5}]:} \PY{k+kn}{import} \PY{n+nn}{matplotlib}\PY{n+nn}{.}\PY{n+nn}{pyplot} \PY{k}{as} \PY{n+nn}{plt}
        
        \PY{k+kn}{from} \PY{n+nn}{skimage}\PY{n+nn}{.}\PY{n+nn}{feature} \PY{k}{import} \PY{n}{hog}
        \PY{k+kn}{from} \PY{n+nn}{skimage} \PY{k}{import} \PY{n}{data}\PY{p}{,} \PY{n}{exposure}
        \PY{k+kn}{from} \PY{n+nn}{skimage}\PY{n+nn}{.}\PY{n+nn}{color} \PY{k}{import} \PY{n}{rgb2gray}
        \PY{k+kn}{from} \PY{n+nn}{skimage}\PY{n+nn}{.}\PY{n+nn}{io} \PY{k}{import} \PY{n}{imread}
        \PY{k+kn}{from} \PY{n+nn}{skimage}\PY{n+nn}{.}\PY{n+nn}{transform} \PY{k}{import} \PY{n}{resize}
        
        \PY{k+kn}{from} \PY{n+nn}{keras}\PY{n+nn}{.}\PY{n+nn}{preprocessing} \PY{k}{import} \PY{n}{image}
        
        \PY{c+c1}{\PYZsh{} Hide all warnings}
        \PY{k+kn}{import} \PY{n+nn}{warnings}
        \PY{n}{warnings}\PY{o}{.}\PY{n}{filterwarnings}\PY{p}{(}\PY{l+s+s1}{\PYZsq{}}\PY{l+s+s1}{ignore}\PY{l+s+s1}{\PYZsq{}}\PY{p}{)}
\end{Verbatim}


    \begin{Verbatim}[commandchars=\\\{\}]
Using TensorFlow backend.

    \end{Verbatim}

    \hypertarget{image-data-representations}{%
\section{ Image data representations}\label{image-data-representations}}

\hypertarget{load-and-visualize-a-sample-image}{%
\subsection{ Load and visualize a sample
image}\label{load-and-visualize-a-sample-image}}

We start by loading an image. Then we do some pre-processing to prepare
the image for feature extraction.

Our goal is to represent images using a fixed-length vector, despite the
original image height and width. Therefore, we resize all images to a
certain height and width.

    \begin{Verbatim}[commandchars=\\\{\}]
{\color{incolor}In [{\color{incolor}6}]:} \PY{k}{def} \PY{n+nf}{center\PYZus{}crop\PYZus{}image}\PY{p}{(}\PY{n}{im}\PY{p}{,} \PY{n}{size}\PY{o}{=}\PY{l+m+mi}{224}\PY{p}{)}\PY{p}{:}
        
            \PY{k}{if} \PY{n}{im}\PY{o}{.}\PY{n}{shape}\PY{p}{[}\PY{l+m+mi}{2}\PY{p}{]} \PY{o}{==} \PY{l+m+mi}{4}\PY{p}{:} \PY{c+c1}{\PYZsh{} Remove the alpha channel}
                \PY{n}{im} \PY{o}{=} \PY{n}{im}\PY{p}{[}\PY{p}{:}\PY{p}{,} \PY{p}{:}\PY{p}{,} \PY{l+m+mi}{0}\PY{p}{:}\PY{l+m+mi}{3}\PY{p}{]}
        
            \PY{c+c1}{\PYZsh{} Resize so smallest dim = 224, preserving aspect ratio}
            \PY{n}{h}\PY{p}{,} \PY{n}{w}\PY{p}{,} \PY{n}{\PYZus{}} \PY{o}{=} \PY{n}{im}\PY{o}{.}\PY{n}{shape}
            \PY{k}{if} \PY{n}{h} \PY{o}{\PYZlt{}} \PY{n}{w}\PY{p}{:}
                \PY{n}{im} \PY{o}{=} \PY{n}{resize}\PY{p}{(}\PY{n}{image}\PY{o}{=}\PY{n}{im}\PY{p}{,} \PY{n}{output\PYZus{}shape}\PY{o}{=}\PY{p}{(}\PY{l+m+mi}{224}\PY{p}{,} \PY{n+nb}{int}\PY{p}{(}\PY{n}{w} \PY{o}{*} \PY{l+m+mi}{224} \PY{o}{/} \PY{n}{h}\PY{p}{)}\PY{p}{)}\PY{p}{)}
            \PY{k}{else}\PY{p}{:}
                \PY{n}{im} \PY{o}{=} \PY{n}{resize}\PY{p}{(}\PY{n}{im}\PY{p}{,} \PY{p}{(}\PY{n+nb}{int}\PY{p}{(}\PY{n}{h} \PY{o}{*} \PY{l+m+mi}{224} \PY{o}{/} \PY{n}{w}\PY{p}{)}\PY{p}{,} \PY{l+m+mi}{224}\PY{p}{)}\PY{p}{)}
        
            \PY{c+c1}{\PYZsh{} Center crop to 224x224}
            \PY{n}{h}\PY{p}{,} \PY{n}{w}\PY{p}{,} \PY{n}{\PYZus{}} \PY{o}{=} \PY{n}{im}\PY{o}{.}\PY{n}{shape}
            \PY{n}{im} \PY{o}{=} \PY{n}{im}\PY{p}{[}\PY{n}{h} \PY{o}{/}\PY{o}{/} \PY{l+m+mi}{2} \PY{o}{\PYZhy{}} \PY{l+m+mi}{112}\PY{p}{:}\PY{n}{h} \PY{o}{/}\PY{o}{/} \PY{l+m+mi}{2} \PY{o}{+} \PY{l+m+mi}{112}\PY{p}{,} \PY{n}{w} \PY{o}{/}\PY{o}{/} \PY{l+m+mi}{2} \PY{o}{\PYZhy{}} \PY{l+m+mi}{112}\PY{p}{:}\PY{n}{w} \PY{o}{/}\PY{o}{/} \PY{l+m+mi}{2} \PY{o}{+} \PY{l+m+mi}{112}\PY{p}{]}
            
            \PY{k}{return} \PY{n}{im}
\end{Verbatim}


    \begin{Verbatim}[commandchars=\\\{\}]
{\color{incolor}In [{\color{incolor}7}]:} \PY{n}{image1}\PY{o}{=}\PY{l+s+s2}{\PYZdq{}}\PY{l+s+s2}{cars.jpg}\PY{l+s+s2}{\PYZdq{}}
        \PY{n}{image2}\PY{o}{=}\PY{l+s+s2}{\PYZdq{}}\PY{l+s+s2}{lena.jpg}\PY{l+s+s2}{\PYZdq{}}
        \PY{n}{image3}\PY{o}{=}\PY{l+s+s2}{\PYZdq{}}\PY{l+s+s2}{street.jpg}\PY{l+s+s2}{\PYZdq{}}
        \PY{n}{image4}\PY{o}{=}\PY{l+s+s2}{\PYZdq{}}\PY{l+s+s2}{bird.jpg}\PY{l+s+s2}{\PYZdq{}}
        \PY{n}{image5}\PY{o}{=}\PY{l+s+s2}{\PYZdq{}}\PY{l+s+s2}{cup.jpg}\PY{l+s+s2}{\PYZdq{}}
        \PY{n}{image6}\PY{o}{=}\PY{l+s+s2}{\PYZdq{}}\PY{l+s+s2}{red.jpg}\PY{l+s+s2}{\PYZdq{}}
        \PY{n}{image7}\PY{o}{=}\PY{l+s+s2}{\PYZdq{}}\PY{l+s+s2}{strawberries.jpg}\PY{l+s+s2}{\PYZdq{}}
        \PY{n}{image8}\PY{o}{=}\PY{l+s+s2}{\PYZdq{}}\PY{l+s+s2}{ferrari.jpg}\PY{l+s+s2}{\PYZdq{}}
        \PY{n}{image9}\PY{o}{=}\PY{l+s+s2}{\PYZdq{}}\PY{l+s+s2}{bird2.jpg}\PY{l+s+s2}{\PYZdq{}}
        \PY{n}{image10}\PY{o}{=}\PY{l+s+s2}{\PYZdq{}}\PY{l+s+s2}{street2.jpg}\PY{l+s+s2}{\PYZdq{}}
        \PY{n}{image11}\PY{o}{=}\PY{l+s+s2}{\PYZdq{}}\PY{l+s+s2}{car2.jpg}\PY{l+s+s2}{\PYZdq{}}
        \PY{n}{image12}\PY{o}{=}\PY{l+s+s2}{\PYZdq{}}\PY{l+s+s2}{fat\PYZus{}bird.jpg}\PY{l+s+s2}{\PYZdq{}}
        \PY{n}{image13}\PY{o}{=}\PY{l+s+s2}{\PYZdq{}}\PY{l+s+s2}{flowers.jpg}\PY{l+s+s2}{\PYZdq{}}
        \PY{n}{image14}\PY{o}{=}\PY{l+s+s2}{\PYZdq{}}\PY{l+s+s2}{flowers2.jpg}\PY{l+s+s2}{\PYZdq{}}
        \PY{n}{image15}\PY{o}{=}\PY{l+s+s2}{\PYZdq{}}\PY{l+s+s2}{red2.jpg}\PY{l+s+s2}{\PYZdq{}}
        
        \PY{n}{img} \PY{o}{=} \PY{n}{imread}\PY{p}{(}\PY{l+s+s2}{\PYZdq{}}\PY{l+s+s2}{data/}\PY{l+s+s2}{\PYZdq{}} \PY{o}{+} \PY{n}{image15}\PY{p}{)}
        \PY{n+nb}{print}\PY{p}{(}\PY{l+s+s2}{\PYZdq{}}\PY{l+s+s2}{Original shape: }\PY{l+s+si}{\PYZob{}\PYZcb{}}\PY{l+s+s2}{\PYZdq{}}\PY{o}{.}\PY{n}{format}\PY{p}{(}\PY{n}{img}\PY{o}{.}\PY{n}{shape}\PY{p}{)}\PY{p}{)}
\end{Verbatim}


    \begin{Verbatim}[commandchars=\\\{\}]
Original shape: (683, 1024, 3)

    \end{Verbatim}

    In order to keep the image aspect ratio, we crop the image at the
center, and resize it:

    \begin{Verbatim}[commandchars=\\\{\}]
{\color{incolor}In [{\color{incolor}8}]:} \PY{c+c1}{\PYZsh{}\PYZsh{} Resize image}
        \PY{c+c1}{\PYZsh{}rgb = resize(rgb, (224, 224))}
        
        \PY{c+c1}{\PYZsh{}\PYZsh{} Resize image with center cropping (square)}
        \PY{n}{img} \PY{o}{=} \PY{n}{center\PYZus{}crop\PYZus{}image}\PY{p}{(}\PY{n}{img}\PY{p}{,} \PY{n}{size}\PY{o}{=}\PY{l+m+mi}{224}\PY{p}{)}
        \PY{n+nb}{print}\PY{p}{(}\PY{l+s+s2}{\PYZdq{}}\PY{l+s+s2}{Final shape: }\PY{l+s+si}{\PYZob{}\PYZcb{}}\PY{l+s+s2}{\PYZdq{}}\PY{o}{.}\PY{n}{format}\PY{p}{(}\PY{n}{img}\PY{o}{.}\PY{n}{shape}\PY{p}{)}\PY{p}{)}
\end{Verbatim}


    \begin{Verbatim}[commandchars=\\\{\}]
Final shape: (224, 224, 3)

    \end{Verbatim}

    For some image features, we do not need color information. Thus, we can
convert the image to grayscale:

    \begin{Verbatim}[commandchars=\\\{\}]
{\color{incolor}In [{\color{incolor}9}]:} \PY{n}{img\PYZus{}gray} \PY{o}{=} \PY{n}{rgb2gray}\PY{p}{(}\PY{n}{img}\PY{p}{)}
        
        \PY{n+nb}{print}\PY{p}{(}\PY{l+s+s2}{\PYZdq{}}\PY{l+s+s2}{Grayscale final shape: }\PY{l+s+si}{\PYZob{}\PYZcb{}}\PY{l+s+s2}{\PYZdq{}}\PY{o}{.}\PY{n}{format}\PY{p}{(}\PY{n}{img\PYZus{}gray}\PY{o}{.}\PY{n}{shape}\PY{p}{)}\PY{p}{)}
\end{Verbatim}


    \begin{Verbatim}[commandchars=\\\{\}]
Grayscale final shape: (224, 224)

    \end{Verbatim}

    Visualize both resized original and grayscale images:

    \begin{Verbatim}[commandchars=\\\{\}]
{\color{incolor}In [{\color{incolor}10}]:} \PY{n}{fig}\PY{p}{,} \PY{p}{(}\PY{n}{ax1}\PY{p}{,} \PY{n}{ax2}\PY{p}{)} \PY{o}{=} \PY{n}{plt}\PY{o}{.}\PY{n}{subplots}\PY{p}{(}\PY{l+m+mi}{1}\PY{p}{,} \PY{l+m+mi}{2}\PY{p}{,} \PY{n}{figsize}\PY{o}{=}\PY{p}{(}\PY{l+m+mi}{10}\PY{p}{,} \PY{l+m+mi}{6}\PY{p}{)}\PY{p}{,} \PY{n}{sharex}\PY{o}{=}\PY{k+kc}{True}\PY{p}{,} \PY{n}{sharey}\PY{o}{=}\PY{k+kc}{True}\PY{p}{)}
         
         \PY{n}{ax1}\PY{o}{.}\PY{n}{axis}\PY{p}{(}\PY{l+s+s1}{\PYZsq{}}\PY{l+s+s1}{off}\PY{l+s+s1}{\PYZsq{}}\PY{p}{)}
         \PY{n}{ax1}\PY{o}{.}\PY{n}{imshow}\PY{p}{(}\PY{n}{img}\PY{p}{)}
         \PY{n}{ax1}\PY{o}{.}\PY{n}{set\PYZus{}title}\PY{p}{(}\PY{l+s+s1}{\PYZsq{}}\PY{l+s+s1}{Input image}\PY{l+s+s1}{\PYZsq{}}\PY{p}{,} \PY{n}{fontsize}\PY{o}{=}\PY{l+m+mi}{20}\PY{p}{)}
         \PY{n}{ax2}\PY{o}{.}\PY{n}{axis}\PY{p}{(}\PY{l+s+s1}{\PYZsq{}}\PY{l+s+s1}{off}\PY{l+s+s1}{\PYZsq{}}\PY{p}{)}
         \PY{n}{ax2}\PY{o}{.}\PY{n}{imshow}\PY{p}{(}\PY{n}{img\PYZus{}gray}\PY{p}{,} \PY{n}{cmap}\PY{o}{=}\PY{n}{plt}\PY{o}{.}\PY{n}{cm}\PY{o}{.}\PY{n}{gray}\PY{p}{)}
         \PY{n}{ax2}\PY{o}{.}\PY{n}{set\PYZus{}title}\PY{p}{(}\PY{l+s+s1}{\PYZsq{}}\PY{l+s+s1}{Grayscale image}\PY{l+s+s1}{\PYZsq{}}\PY{p}{,} \PY{n}{fontsize}\PY{o}{=}\PY{l+m+mi}{20}\PY{p}{)}
         \PY{n}{plt}\PY{o}{.}\PY{n}{show}\PY{p}{(}\PY{p}{)}
\end{Verbatim}


    \begin{center}
    \adjustimage{max size={0.9\linewidth}{0.9\paperheight}}{output_10_0.png}
    \end{center}
    { \hspace*{\fill} \\}
    
    \hypertarget{histogram-of-oriented-gradients}{%
\subsection{ Histogram of Oriented
Gradients}\label{histogram-of-oriented-gradients}}

    \hypertarget{computing-histogram-of-oriented-gradients-hog-using-skimage-library.-hog-scikit-image-documentation}{%
\paragraph{\texorpdfstring{Computing Histogram of Oriented Gradients
(HoG) using skimage library.
\href{http://scikit-image.org/docs/0.14.x/api/skimage.feature.html\#skimage.feature.hog}{HoG
scikit-image
Documentation}}{Computing Histogram of Oriented Gradients (HoG) using skimage library. HoG scikit-image Documentation}}\label{computing-histogram-of-oriented-gradients-hog-using-skimage-library.-hog-scikit-image-documentation}}

In the HOG feature descriptor, the distribution ( histograms ) of
directions of gradients ( oriented gradients ) are used as features.
Gradients ( x and y derivatives ) of an image are useful as the
magnitude of gradients is large around edges and corners ( regions of
abrupt intensity changes ) and we know that edges and corners provide
more information regarding object shapes than flat regions.

More about HoG descriptor: *
http://mccormickml.com/2013/05/09/hog-person-detector-tutorial/ *
http://mccormickml.com/2013/05/07/gradient-vectors/

    \begin{Verbatim}[commandchars=\\\{\}]
{\color{incolor}In [{\color{incolor}11}]:} \PY{n}{fd}\PY{p}{,} \PY{n}{hog\PYZus{}image} \PY{o}{=} \PY{n}{hog}\PY{p}{(}\PY{n}{img\PYZus{}gray}\PY{p}{,} \PY{n}{orientations}\PY{o}{=}\PY{l+m+mi}{8}\PY{p}{,} \PY{n}{pixels\PYZus{}per\PYZus{}cell}\PY{o}{=}\PY{p}{(}\PY{l+m+mi}{16}\PY{p}{,} \PY{l+m+mi}{16}\PY{p}{)}\PY{p}{,} \PY{n}{visualise}\PY{o}{=}\PY{k+kc}{True}\PY{p}{)}
         \PY{n+nb}{print}\PY{p}{(}\PY{l+s+s2}{\PYZdq{}}\PY{l+s+s2}{HoG Feature vector shape: }\PY{l+s+si}{\PYZob{}\PYZcb{}}\PY{l+s+s2}{\PYZdq{}}\PY{o}{.}\PY{n}{format}\PY{p}{(}\PY{n}{fd}\PY{o}{.}\PY{n}{shape}\PY{p}{)}\PY{p}{)}
\end{Verbatim}


    \begin{Verbatim}[commandchars=\\\{\}]
HoG Feature vector shape: (10368,)

    \end{Verbatim}

    Visualizing the computed HoG feature:

    \begin{Verbatim}[commandchars=\\\{\}]
{\color{incolor}In [{\color{incolor}12}]:} \PY{n}{fig}\PY{p}{,} \PY{p}{(}\PY{n}{ax1}\PY{p}{,} \PY{n}{ax2}\PY{p}{)} \PY{o}{=} \PY{n}{plt}\PY{o}{.}\PY{n}{subplots}\PY{p}{(}\PY{l+m+mi}{1}\PY{p}{,} \PY{l+m+mi}{2}\PY{p}{,} \PY{n}{figsize}\PY{o}{=}\PY{p}{(}\PY{l+m+mi}{10}\PY{p}{,} \PY{l+m+mi}{6}\PY{p}{)}\PY{p}{,} \PY{n}{sharex}\PY{o}{=}\PY{k+kc}{True}\PY{p}{,} \PY{n}{sharey}\PY{o}{=}\PY{k+kc}{True}\PY{p}{)}
         
         \PY{n}{ax1}\PY{o}{.}\PY{n}{axis}\PY{p}{(}\PY{l+s+s1}{\PYZsq{}}\PY{l+s+s1}{off}\PY{l+s+s1}{\PYZsq{}}\PY{p}{)}
         \PY{n}{ax1}\PY{o}{.}\PY{n}{imshow}\PY{p}{(}\PY{n}{img\PYZus{}gray}\PY{p}{,} \PY{n}{cmap}\PY{o}{=}\PY{n}{plt}\PY{o}{.}\PY{n}{cm}\PY{o}{.}\PY{n}{gray}\PY{p}{)}
         \PY{n}{ax1}\PY{o}{.}\PY{n}{set\PYZus{}title}\PY{p}{(}\PY{l+s+s1}{\PYZsq{}}\PY{l+s+s1}{Input image}\PY{l+s+s1}{\PYZsq{}}\PY{p}{,} \PY{n}{fontsize}\PY{o}{=}\PY{l+m+mi}{20}\PY{p}{)}
         
         \PY{c+c1}{\PYZsh{} Rescale histogram for better display}
         \PY{n}{hog\PYZus{}image\PYZus{}rescaled} \PY{o}{=} \PY{n}{exposure}\PY{o}{.}\PY{n}{rescale\PYZus{}intensity}\PY{p}{(}\PY{n}{hog\PYZus{}image}\PY{p}{,} \PY{n}{in\PYZus{}range}\PY{o}{=}\PY{p}{(}\PY{l+m+mi}{0}\PY{p}{,} \PY{l+m+mi}{10}\PY{p}{)}\PY{p}{)}
         
         \PY{n}{ax2}\PY{o}{.}\PY{n}{axis}\PY{p}{(}\PY{l+s+s1}{\PYZsq{}}\PY{l+s+s1}{off}\PY{l+s+s1}{\PYZsq{}}\PY{p}{)}
         \PY{n}{ax2}\PY{o}{.}\PY{n}{imshow}\PY{p}{(}\PY{n}{hog\PYZus{}image\PYZus{}rescaled}\PY{p}{,} \PY{n}{cmap}\PY{o}{=}\PY{n}{plt}\PY{o}{.}\PY{n}{cm}\PY{o}{.}\PY{n}{gray}\PY{p}{)}
         \PY{n}{ax2}\PY{o}{.}\PY{n}{set\PYZus{}title}\PY{p}{(}\PY{l+s+s1}{\PYZsq{}}\PY{l+s+s1}{Histogram of Oriented Gradients}\PY{l+s+s1}{\PYZsq{}}\PY{p}{,} \PY{n}{fontsize}\PY{o}{=}\PY{l+m+mi}{18}\PY{p}{)}
         \PY{n}{plt}\PY{o}{.}\PY{n}{show}\PY{p}{(}\PY{p}{)}
\end{Verbatim}


    \begin{center}
    \adjustimage{max size={0.9\linewidth}{0.9\paperheight}}{output_15_0.png}
    \end{center}
    { \hspace*{\fill} \\}
    
    \hypertarget{histogram-of-color}{%
\subsection{ Histogram of Color}\label{histogram-of-color}}

    \begin{Verbatim}[commandchars=\\\{\}]
{\color{incolor}In [{\color{incolor}13}]:} \PY{c+c1}{\PYZsh{} Straight forward HoC implementation on RGB space}
         \PY{c+c1}{\PYZsh{} For a more complete implementation, with better parametrization, etc., you can check the OpenCV library.}
         
         \PY{k+kn}{import} \PY{n+nn}{matplotlib}\PY{n+nn}{.}\PY{n+nn}{pyplot} \PY{k}{as} \PY{n+nn}{plt}
         \PY{k+kn}{import} \PY{n+nn}{numpy} \PY{k}{as} \PY{n+nn}{np}
         
         \PY{k}{def} \PY{n+nf}{hoc}\PY{p}{(}\PY{n}{im}\PY{p}{,} \PY{n}{bins}\PY{o}{=}\PY{p}{(}\PY{l+m+mi}{16}\PY{p}{,}\PY{l+m+mi}{16}\PY{p}{,}\PY{l+m+mi}{16}\PY{p}{)}\PY{p}{,} \PY{n}{hist\PYZus{}range}\PY{o}{=}\PY{p}{(}\PY{l+m+mi}{256}\PY{p}{,} \PY{l+m+mi}{256}\PY{p}{,} \PY{l+m+mi}{256}\PY{p}{)}\PY{p}{)}\PY{p}{:}
             \PY{n}{im\PYZus{}r} \PY{o}{=} \PY{n}{im}\PY{p}{[}\PY{p}{:}\PY{p}{,}\PY{p}{:}\PY{p}{,}\PY{l+m+mi}{0}\PY{p}{]}
             \PY{n}{im\PYZus{}g} \PY{o}{=} \PY{n}{im}\PY{p}{[}\PY{p}{:}\PY{p}{,}\PY{p}{:}\PY{p}{,}\PY{l+m+mi}{1}\PY{p}{]}
             \PY{n}{im\PYZus{}b} \PY{o}{=} \PY{n}{im}\PY{p}{[}\PY{p}{:}\PY{p}{,}\PY{p}{:}\PY{p}{,}\PY{l+m+mi}{2}\PY{p}{]}
             
             \PY{n}{red\PYZus{}level} \PY{o}{=} \PY{n}{hist\PYZus{}range}\PY{p}{[}\PY{l+m+mi}{0}\PY{p}{]} \PY{o}{/} \PY{n}{bins}\PY{p}{[}\PY{l+m+mi}{0}\PY{p}{]}
             \PY{n}{green\PYZus{}level} \PY{o}{=} \PY{n}{hist\PYZus{}range}\PY{p}{[}\PY{l+m+mi}{1}\PY{p}{]} \PY{o}{/} \PY{n}{bins}\PY{p}{[}\PY{l+m+mi}{1}\PY{p}{]}
             \PY{n}{blue\PYZus{}level} \PY{o}{=} \PY{n}{hist\PYZus{}range}\PY{p}{[}\PY{l+m+mi}{2}\PY{p}{]} \PY{o}{/} \PY{n}{bins}\PY{p}{[}\PY{l+m+mi}{2}\PY{p}{]}
             
             \PY{n}{im\PYZus{}red\PYZus{}levels} \PY{o}{=} \PY{n}{im\PYZus{}r} \PY{o}{/} \PY{n}{red\PYZus{}level}
             \PY{n}{im\PYZus{}green\PYZus{}levels} \PY{o}{=} \PY{n}{im\PYZus{}g} \PY{o}{/} \PY{n}{green\PYZus{}level}
             \PY{n}{im\PYZus{}blue\PYZus{}levels} \PY{o}{=} \PY{n}{im\PYZus{}b} \PY{o}{/} \PY{n}{blue\PYZus{}level}
             
             \PY{n}{ind} \PY{o}{=} \PY{n}{im\PYZus{}blue\PYZus{}levels}\PY{o}{*}\PY{n}{bins}\PY{p}{[}\PY{l+m+mi}{0}\PY{p}{]}\PY{o}{*}\PY{n}{bins}\PY{p}{[}\PY{l+m+mi}{1}\PY{p}{]}\PY{o}{+} \PY{n}{im\PYZus{}green\PYZus{}levels}\PY{o}{*}\PY{n}{bins}\PY{p}{[}\PY{l+m+mi}{0}\PY{p}{]} \PY{o}{+} \PY{n}{im\PYZus{}red\PYZus{}levels}
             
             \PY{n}{hist\PYZus{}r}\PY{p}{,} \PY{n}{bins\PYZus{}r} \PY{o}{=} \PY{n}{np}\PY{o}{.}\PY{n}{histogram}\PY{p}{(}\PY{n}{ind}\PY{o}{.}\PY{n}{flatten}\PY{p}{(}\PY{p}{)}\PY{p}{,} \PY{n}{bins}\PY{p}{[}\PY{l+m+mi}{0}\PY{p}{]}\PY{o}{*}\PY{n}{bins}\PY{p}{[}\PY{l+m+mi}{1}\PY{p}{]}\PY{o}{*}\PY{n}{bins}\PY{p}{[}\PY{l+m+mi}{2}\PY{p}{]}\PY{p}{)}
             
             \PY{k}{return} \PY{n}{hist\PYZus{}r}\PY{p}{,} \PY{n}{bins\PYZus{}r}
\end{Verbatim}


    \begin{Verbatim}[commandchars=\\\{\}]
{\color{incolor}In [{\color{incolor}20}]:} \PY{k+kn}{from} \PY{n+nn}{skimage} \PY{k}{import} \PY{n}{img\PYZus{}as\PYZus{}ubyte}
         
         \PY{c+c1}{\PYZsh{} convert image pixels to [0, 255] range, and to uint8 type }
         \PY{n}{img\PYZus{}int} \PY{o}{=} \PY{n}{img\PYZus{}as\PYZus{}ubyte}\PY{p}{(}\PY{n}{img}\PY{p}{)}
         
         \PY{c+c1}{\PYZsh{} Set bins range according to the image color space}
         \PY{n}{hist\PYZus{}range\PYZus{}hsv}\PY{o}{=}\PY{p}{(}\PY{l+m+mi}{180}\PY{p}{,} \PY{l+m+mi}{256}\PY{p}{,} \PY{l+m+mi}{256}\PY{p}{)}
         \PY{n}{hist\PYZus{}range\PYZus{}rgb}\PY{o}{=}\PY{p}{(}\PY{l+m+mi}{256}\PY{p}{,} \PY{l+m+mi}{256}\PY{p}{,} \PY{l+m+mi}{256}\PY{p}{)}
         \PY{n}{hist\PYZus{}range} \PY{o}{=} \PY{n}{hist\PYZus{}range\PYZus{}hsv}
         
         \PY{c+c1}{\PYZsh{} Compute histogram}
         \PY{n}{hist}\PY{p}{,} \PY{n}{bin\PYZus{}edges} \PY{o}{=} \PY{n}{hoc}\PY{p}{(}\PY{n}{img\PYZus{}int}\PY{p}{,} \PY{n}{bins}\PY{o}{=}\PY{p}{(}\PY{l+m+mi}{4}\PY{p}{,}\PY{l+m+mi}{4}\PY{p}{,}\PY{l+m+mi}{4}\PY{p}{)}\PY{p}{,} \PY{n}{hist\PYZus{}range}\PY{o}{=}\PY{n}{hist\PYZus{}range}\PY{p}{)}
         \PY{n+nb}{print}\PY{p}{(}\PY{l+s+s2}{\PYZdq{}}\PY{l+s+s2}{Resulting HoC vector shape: }\PY{l+s+si}{\PYZob{}\PYZcb{}}\PY{l+s+s2}{\PYZdq{}}\PY{o}{.}\PY{n}{format}\PY{p}{(}\PY{n}{hist}\PY{o}{.}\PY{n}{shape}\PY{p}{)}\PY{p}{)}
\end{Verbatim}


    \begin{Verbatim}[commandchars=\\\{\}]
Resulting HoC vector shape: (64,)

    \end{Verbatim}

    \hypertarget{visualize-the-resulting-histogram}{%
\paragraph{Visualize the resulting
histogram}\label{visualize-the-resulting-histogram}}

    \begin{Verbatim}[commandchars=\\\{\}]
{\color{incolor}In [{\color{incolor}21}]:} \PY{n}{fig} \PY{o}{=} \PY{n}{plt}\PY{o}{.}\PY{n}{figure}\PY{p}{(}\PY{n}{figsize}\PY{o}{=}\PY{p}{(}\PY{l+m+mi}{10}\PY{p}{,}\PY{l+m+mi}{6}\PY{p}{)}\PY{p}{)}
         \PY{n}{ax} \PY{o}{=} \PY{n}{fig}\PY{o}{.}\PY{n}{add\PYZus{}subplot}\PY{p}{(}\PY{l+m+mi}{111}\PY{p}{)}
         \PY{n}{ax}\PY{o}{.}\PY{n}{bar}\PY{p}{(}\PY{n}{bin\PYZus{}edges}\PY{p}{[}\PY{p}{:}\PY{o}{\PYZhy{}}\PY{l+m+mi}{1}\PY{p}{]}\PY{p}{,} \PY{n}{hist} \PY{p}{,}\PY{n}{width}\PY{o}{=}\PY{l+m+mi}{1}\PY{p}{)}
         \PY{n}{ax}\PY{o}{.}\PY{n}{set\PYZus{}xticks}\PY{p}{(}\PY{p}{[}\PY{p}{]}\PY{p}{)}
         \PY{n}{ax}\PY{o}{.}\PY{n}{set\PYZus{}xlim}\PY{p}{(}\PY{n}{bin\PYZus{}edges}\PY{p}{[}\PY{p}{:}\PY{o}{\PYZhy{}}\PY{l+m+mi}{1}\PY{p}{]}\PY{o}{.}\PY{n}{min}\PY{p}{(}\PY{p}{)}\PY{o}{*}\PY{o}{\PYZhy{}}\PY{l+m+mi}{2}\PY{p}{,} \PY{n+nb}{max}\PY{p}{(}\PY{n}{bin\PYZus{}edges}\PY{o}{.}\PY{n}{max}\PY{p}{(}\PY{p}{)}\PY{p}{,} \PY{n}{hist}\PY{o}{.}\PY{n}{shape}\PY{p}{[}\PY{l+m+mi}{0}\PY{p}{]}\PY{o}{*}\PY{l+m+mf}{1.3}\PY{p}{)}\PY{p}{)}
         \PY{n}{plt}\PY{o}{.}\PY{n}{show}\PY{p}{(}\PY{p}{)}
\end{Verbatim}


    \begin{center}
    \adjustimage{max size={0.9\linewidth}{0.9\paperheight}}{output_20_0.png}
    \end{center}
    { \hspace*{\fill} \\}
    
    \hypertarget{text-data-representations}{%
\section{ Text data representations}\label{text-data-representations}}

\hypertarget{corpus}{%
\subsection{ Corpus}\label{corpus}}

We will use the brown corpus from NLTK.

    \begin{Verbatim}[commandchars=\\\{\}]
{\color{incolor}In [{\color{incolor}12}]:} \PY{k+kn}{import} \PY{n+nn}{nltk}
         \PY{k+kn}{from} \PY{n+nn}{nltk}\PY{n+nn}{.}\PY{n+nn}{corpus} \PY{k}{import} \PY{n}{brown}
         \PY{n}{nltk}\PY{o}{.}\PY{n}{download}\PY{p}{(}\PY{l+s+s1}{\PYZsq{}}\PY{l+s+s1}{brown}\PY{l+s+s1}{\PYZsq{}}\PY{p}{)}
         \PY{n}{corpus} \PY{o}{=} \PY{n}{brown}\PY{o}{.}\PY{n}{sents}\PY{p}{(}\PY{p}{)}
         
         \PY{n}{corpus} \PY{o}{=} \PY{p}{[}\PY{l+s+s2}{\PYZdq{}}\PY{l+s+s2}{ }\PY{l+s+s2}{\PYZdq{}}\PY{o}{.}\PY{n}{join}\PY{p}{(}\PY{n}{sent}\PY{p}{)} \PY{k}{for} \PY{n}{sent} \PY{o+ow}{in} \PY{n}{corpus}\PY{p}{]}
         \PY{n+nb}{print}\PY{p}{(}\PY{l+s+s2}{\PYZdq{}}\PY{l+s+s2}{Total texts: }\PY{l+s+si}{\PYZob{}\PYZcb{}}\PY{l+s+s2}{\PYZdq{}}\PY{o}{.}\PY{n}{format}\PY{p}{(}\PY{n+nb}{len}\PY{p}{(}\PY{n}{corpus}\PY{p}{)}\PY{p}{)}\PY{p}{)}
\end{Verbatim}


    \begin{Verbatim}[commandchars=\\\{\}]
[nltk\_data] Downloading package brown to
[nltk\_data]     C:\textbackslash{}Users\textbackslash{}Joao\textbackslash{}AppData\textbackslash{}Roaming\textbackslash{}nltk\_data{\ldots}
[nltk\_data]   Package brown is already up-to-date!
Total texts: 57340

    \end{Verbatim}

    \begin{Verbatim}[commandchars=\\\{\}]
{\color{incolor}In [{\color{incolor}13}]:} \PY{c+c1}{\PYZsh{} Let\PYZsq{}s inspect some documents from the corpus}
         \PY{n+nb}{print}\PY{p}{(}\PY{l+s+s2}{\PYZdq{}}\PY{l+s+s2}{Doc 1: }\PY{l+s+si}{\PYZob{}\PYZcb{}}\PY{l+s+se}{\PYZbs{}n}\PY{l+s+s2}{Doc 2: }\PY{l+s+si}{\PYZob{}\PYZcb{}}\PY{l+s+s2}{\PYZdq{}}\PY{o}{.}\PY{n}{format}\PY{p}{(}\PY{n}{corpus}\PY{p}{[}\PY{l+m+mi}{2555}\PY{p}{]}\PY{p}{,} \PY{n}{corpus}\PY{p}{[}\PY{l+m+mi}{1112}\PY{p}{]}\PY{p}{)}\PY{p}{)}
\end{Verbatim}


    \begin{Verbatim}[commandchars=\\\{\}]
Doc 1: Buchheister pledged the land would be an `` inviolate '' sanctuary for all birds , animals and plants .
Doc 2: She snapped five tenths of a second off the mark set by Helen Shipley , of Wellsley College , in the National A.A.U. meet in Columbus , Ohio .

    \end{Verbatim}

    \hypertarget{bag-of-words}{%
\subsection{ Bag-of-words}\label{bag-of-words}}

In the Bag of Words representation, documents are represented by a
numerical sparse vector, whose dimension corresponds to the vocabulary
size of the whole collection. Then, the set of all documents can be
efficiently represented by a sparse matrix.

To illustrate this, consider the following corpus with two documents:

\begin{verbatim}
corpus = [
            ["some random sentence"],  # doc1
            ["another sentence random sentence"] # doc2
         ]
\end{verbatim}

The vocabulary size of our corpus is 4 (there are 4 different words).
Thus, each document will be represented as a 4-dimensional numeric
vector. Namely, assuming that we sort words alphabetically, we obtain:

\begin{verbatim}
        another, random, sentence, some
doc1 = [   0,       1,      1,       1  ]
doc1 = [   1,       1,      2,       0  ] 
\end{verbatim}

Each entry denotes the counts of each word, in the document. Notice that
we lose information regarding word order, in the BoW representation.

Using counts to create the BoW representation is among the most simple
representations. There are alternative ways to do it which you can then
explore such as TF-IDF, and others. For now, we will stick with BoW
representation based on counts.

\hypertarget{extracting-bow-vectors}{%
\subsubsection{Extracting BoW vectors}\label{extracting-bow-vectors}}

We will use the CountVectorizer class. Check the documentation to
understand the API and the default values of the supported parameters:
http://scikit-learn.org/stable/modules/generated/sklearn.feature\_extraction.text.CountVectorizer.html

The default tokenizer splits documents by whitespaces. This is far from
optimal for Web text, but is a good starting point.

\hypertarget{using-a-tokenizer-from-nltk}{%
\subsubsection{Using a tokenizer from
NLTK}\label{using-a-tokenizer-from-nltk}}

When dealing with text gathered from the Web, it is crucial to have a
good tokenizer. To understand why, inspect some of the documents from
the brown corpus.

We will use one of the tokenizers available in NLTK together with the
CountVectorizer class.

    \begin{Verbatim}[commandchars=\\\{\}]
{\color{incolor}In [{\color{incolor}14}]:} \PY{c+c1}{\PYZsh{} NLTK Tokenizers documentation: https://www.nltk.org/api/nltk.tokenize.html}
         
         \PY{c+c1}{\PYZsh{} from nltk.tokenize import TweetTokenizer}
         \PY{c+c1}{\PYZsh{} tknzr = TweetTokenizer()}
         
         \PY{c+c1}{\PYZsh{} Instantiate TreebankWordTokenizer}
         \PY{k+kn}{from} \PY{n+nn}{nltk}\PY{n+nn}{.}\PY{n+nn}{tokenize} \PY{k}{import} \PY{n}{TreebankWordTokenizer}
         \PY{n}{tknzr} \PY{o}{=} \PY{n}{TreebankWordTokenizer}\PY{p}{(}\PY{p}{)}
\end{Verbatim}


    Create a CountVectorizer object and obtain the BoW sparse matrix for the
corpus:

    \begin{Verbatim}[commandchars=\\\{\}]
{\color{incolor}In [{\color{incolor}15}]:} \PY{k+kn}{from} \PY{n+nn}{sklearn}\PY{n+nn}{.}\PY{n+nn}{feature\PYZus{}extraction}\PY{n+nn}{.}\PY{n+nn}{text} \PY{k}{import} \PY{n}{CountVectorizer}
         
         \PY{n}{vectorizer} \PY{o}{=} \PY{n}{CountVectorizer}\PY{p}{(}\PY{n}{tokenizer}\PY{o}{=}\PY{n}{tknzr}\PY{o}{.}\PY{n}{tokenize}\PY{p}{)}
         \PY{n}{texts\PYZus{}bow} \PY{o}{=} \PY{n}{vectorizer}\PY{o}{.}\PY{n}{fit\PYZus{}transform}\PY{p}{(}\PY{n}{corpus}\PY{p}{)}
         \PY{n+nb}{print}\PY{p}{(}\PY{l+s+s2}{\PYZdq{}}\PY{l+s+s2}{texts\PYZus{}bow shape: }\PY{l+s+si}{\PYZob{}\PYZcb{}}\PY{l+s+s2}{ \PYZhy{} Type: }\PY{l+s+si}{\PYZob{}\PYZcb{}}\PY{l+s+s2}{\PYZdq{}}\PY{o}{.}\PY{n}{format}\PY{p}{(}\PY{n}{texts\PYZus{}bow}\PY{o}{.}\PY{n}{shape}\PY{p}{,} \PY{n+nb}{type}\PY{p}{(}\PY{n}{texts\PYZus{}bow}\PY{p}{)}\PY{p}{)}\PY{p}{)}
\end{Verbatim}


    \begin{Verbatim}[commandchars=\\\{\}]
texts\_bow shape: (57340, 47623) - Type: <class 'scipy.sparse.csr.csr\_matrix'>

    \end{Verbatim}

    The \texttt{texts\_bow} variable is a sparse matrix. You can read more
about sparse matrix (csr\_matrix) here:
https://docs.scipy.org/doc/scipy/reference/generated/scipy.sparse.csr\_matrix.html

Let's inspect the vocabulary:

    \begin{Verbatim}[commandchars=\\\{\}]
{\color{incolor}In [{\color{incolor}16}]:} \PY{n}{vocabulary} \PY{o}{=} \PY{n}{vectorizer}\PY{o}{.}\PY{n}{vocabulary\PYZus{}}
         \PY{n+nb}{print}\PY{p}{(}\PY{l+s+s2}{\PYZdq{}}\PY{l+s+s2}{Vocabulary size: }\PY{l+s+si}{\PYZob{}\PYZcb{}}\PY{l+s+s2}{\PYZdq{}}\PY{o}{.}\PY{n}{format}\PY{p}{(}\PY{n+nb}{len}\PY{p}{(}\PY{n}{vocabulary}\PY{p}{)}\PY{p}{)}\PY{p}{)}
\end{Verbatim}


    \begin{Verbatim}[commandchars=\\\{\}]
Vocabulary size: 47623

    \end{Verbatim}

    The vocabulary size is huge and contains a lot of uninformative words.
Let's do some pre-processing to reduce the vocabulary (e.g.~remove
Stopwords, remove rare words, etc.):

    \begin{Verbatim}[commandchars=\\\{\}]
{\color{incolor}In [{\color{incolor}17}]:} \PY{c+c1}{\PYZsh{} remove stopwords, exclude words that appear less than 3 times}
         \PY{c+c1}{\PYZsh{} Notice that the NLTK tokenizer is passed as parameter to CountVectorizer}
         \PY{n}{vectorizer} \PY{o}{=} \PY{n}{CountVectorizer}\PY{p}{(}\PY{n}{stop\PYZus{}words}\PY{o}{=}\PY{l+s+s2}{\PYZdq{}}\PY{l+s+s2}{english}\PY{l+s+s2}{\PYZdq{}}\PY{p}{,} \PY{n}{min\PYZus{}df}\PY{o}{=}\PY{l+m+mi}{3}\PY{p}{,} \PY{n}{binary}\PY{o}{=}\PY{k+kc}{False}\PY{p}{,} \PY{n}{tokenizer}\PY{o}{=}\PY{n}{tknzr}\PY{o}{.}\PY{n}{tokenize}\PY{p}{)}
         \PY{n}{texts\PYZus{}bow} \PY{o}{=} \PY{n}{vectorizer}\PY{o}{.}\PY{n}{fit\PYZus{}transform}\PY{p}{(}\PY{n}{corpus}\PY{p}{)}
         \PY{n}{vocabulary} \PY{o}{=} \PY{n}{vectorizer}\PY{o}{.}\PY{n}{vocabulary\PYZus{}}
         \PY{n+nb}{print}\PY{p}{(}\PY{l+s+s2}{\PYZdq{}}\PY{l+s+s2}{Vocabulary size: }\PY{l+s+si}{\PYZob{}\PYZcb{}}\PY{l+s+s2}{\PYZdq{}}\PY{o}{.}\PY{n}{format}\PY{p}{(}\PY{n+nb}{len}\PY{p}{(}\PY{n}{vocabulary}\PY{p}{)}\PY{p}{)}\PY{p}{)}
\end{Verbatim}


    \begin{Verbatim}[commandchars=\\\{\}]
Vocabulary size: 19789

    \end{Verbatim}

    Using these two pre-processing steps, it gets much better.

You can check CountVectorizer documentation for more pre-processing
steps (e.g.~lowercasing, better tokenizer, stemming, etc.) that may be
adequate for your use-case.

    \hypertarget{searching-on-a-single-feature-space}{%
\section{ Searching on a single feature
space}\label{searching-on-a-single-feature-space}}

\hypertarget{distance-functions-and-normalization}{%
\subsection{ Distance functions and
normalization}\label{distance-functions-and-normalization}}

Let's define a function that takes as input a query vector, and a matrix
of target vectors (the `database'), and returns the K-nearest
neighbours. Make sure you understand the function below.

    \begin{Verbatim}[commandchars=\\\{\}]
{\color{incolor}In [{\color{incolor}18}]:} \PY{k+kn}{from} \PY{n+nn}{sklearn}\PY{n+nn}{.}\PY{n+nn}{metrics} \PY{k}{import} \PY{n}{pairwise\PYZus{}distances}
         
         \PY{k}{def} \PY{n+nf}{k\PYZus{}neighbours}\PY{p}{(}\PY{n}{q}\PY{p}{,} \PY{n}{X}\PY{p}{,} \PY{n}{metric}\PY{o}{=}\PY{l+s+s2}{\PYZdq{}}\PY{l+s+s2}{euclidean}\PY{l+s+s2}{\PYZdq{}}\PY{p}{,} \PY{n}{k}\PY{o}{=}\PY{l+m+mi}{10}\PY{p}{)}\PY{p}{:}
             \PY{c+c1}{\PYZsh{} Check pairwise\PYZus{}distances function docs: http://scikit\PYZhy{}learn.org/stable/modules/generated/sklearn.metrics.pairwise\PYZus{}distances.html\PYZsh{}sklearn.metrics.pairwise\PYZus{}distances}
             \PY{n}{dists} \PY{o}{=} \PY{n}{pairwise\PYZus{}distances}\PY{p}{(}\PY{n}{q}\PY{p}{,} \PY{n}{X}\PY{p}{,} \PY{n}{metric}\PY{o}{=}\PY{n}{metric}\PY{p}{)}
             
             \PY{c+c1}{\PYZsh{} Dists gets a shape 1 x NumDocs. Convert it to shape NumDocs (i.e. drop the first dimension)}
             \PY{n}{dists} \PY{o}{=} \PY{n}{np}\PY{o}{.}\PY{n}{squeeze}\PY{p}{(}\PY{n}{dists}\PY{p}{)}
             \PY{n}{sorted\PYZus{}indexes} \PY{o}{=} \PY{n}{np}\PY{o}{.}\PY{n}{argsort}\PY{p}{(}\PY{n}{dists}\PY{p}{)}
             
             \PY{k}{return} \PY{n}{sorted\PYZus{}indexes}\PY{p}{[}\PY{p}{:}\PY{n}{k}\PY{p}{]}\PY{p}{,} \PY{n}{dists}\PY{p}{[}\PY{n}{sorted\PYZus{}indexes}\PY{p}{[}\PY{p}{:}\PY{n}{k}\PY{p}{]}\PY{p}{]}
\end{Verbatim}


    \hypertarget{normalize-vectors-to-unit-length}{%
\paragraph{Normalize vectors to unit
length:}\label{normalize-vectors-to-unit-length}}

    \begin{Verbatim}[commandchars=\\\{\}]
{\color{incolor}In [{\color{incolor}19}]:} \PY{k+kn}{from} \PY{n+nn}{sklearn}\PY{n+nn}{.}\PY{n+nn}{preprocessing} \PY{k}{import} \PY{n}{normalize}
         \PY{n}{texts\PYZus{}bow} \PY{o}{=} \PY{n}{normalize}\PY{p}{(}\PY{n}{texts\PYZus{}bow}\PY{p}{,} \PY{n}{norm}\PY{o}{=}\PY{l+s+s2}{\PYZdq{}}\PY{l+s+s2}{l2}\PY{l+s+s2}{\PYZdq{}}\PY{p}{)}
\end{Verbatim}


    \hypertarget{bag-of-words}{%
\subsection{ Bag of words}\label{bag-of-words}}

In order to match queries to documents, it is required to apply the same
processing steps to both the query and the documents. Prepare the text
query:

    \begin{Verbatim}[commandchars=\\\{\}]
{\color{incolor}In [{\color{incolor}20}]:} \PY{n}{query1} \PY{o}{=} \PY{l+s+s2}{\PYZdq{}}\PY{l+s+s2}{Lawyer of Boston Yankees in the National Football League}\PY{l+s+s2}{\PYZdq{}}
         \PY{n}{query2} \PY{o}{=} \PY{l+s+s2}{\PYZdq{}}\PY{l+s+s2}{president kennedy}\PY{l+s+s2}{\PYZdq{}}
         \PY{n}{query}\PY{o}{=}\PY{n}{query1}
         
         \PY{c+c1}{\PYZsh{} Transform query in a BoW representation}
         \PY{n}{query\PYZus{}bow} \PY{o}{=} \PY{n}{vectorizer}\PY{o}{.}\PY{n}{transform}\PY{p}{(}\PY{p}{[}\PY{n}{query}\PY{p}{]}\PY{p}{)}
         \PY{n}{query\PYZus{}bow} \PY{o}{=} \PY{n}{normalize}\PY{p}{(}\PY{n}{query\PYZus{}bow}\PY{p}{,} \PY{n}{norm}\PY{o}{=}\PY{l+s+s2}{\PYZdq{}}\PY{l+s+s2}{l2}\PY{l+s+s2}{\PYZdq{}}\PY{p}{)}
\end{Verbatim}


    \hypertarget{query-the-database-using-the-prepared-query}{%
\paragraph{Query the database using the prepared
query:}\label{query-the-database-using-the-prepared-query}}

    The metric that we will use will be the cosine distance.

The function \texttt{pairwise\_distances} computes distances, not
similarities. Thus, it will compute the cosine distance \(1-cos(x,y)\).
Note that this is not a proper metric, as it lacks the triangle
inequality property.

    \begin{Verbatim}[commandchars=\\\{\}]
{\color{incolor}In [{\color{incolor}21}]:} \PY{n}{k\PYZus{}nearest\PYZus{}indexes}\PY{p}{,} \PY{n}{k\PYZus{}nearest\PYZus{}dists} \PY{o}{=} \PY{n}{k\PYZus{}neighbours}\PY{p}{(}\PY{n}{q}\PY{o}{=}\PY{n}{query\PYZus{}bow}\PY{p}{,} \PY{n}{X}\PY{o}{=}\PY{n}{texts\PYZus{}bow}\PY{p}{,} \PY{n}{metric}\PY{o}{=}\PY{l+s+s2}{\PYZdq{}}\PY{l+s+s2}{cosine}\PY{l+s+s2}{\PYZdq{}}\PY{p}{,} \PY{n}{k}\PY{o}{=}\PY{l+m+mi}{10}\PY{p}{)}
\end{Verbatim}


    \begin{Verbatim}[commandchars=\\\{\}]
{\color{incolor}In [{\color{incolor}22}]:} \PY{c+c1}{\PYZsh{} Inspecting the top\PYZhy{}k results: list of tuples (index of document, distance to query, text of document)}
         \PY{n+nb}{list}\PY{p}{(}\PY{n+nb}{zip}\PY{p}{(}\PY{n}{k\PYZus{}nearest\PYZus{}indexes}\PY{p}{,}\PY{n}{k\PYZus{}nearest\PYZus{}dists}\PY{p}{,} \PY{p}{[}\PY{n}{corpus}\PY{p}{[}\PY{n}{i}\PY{p}{]} \PY{k}{for} \PY{n}{i} \PY{o+ow}{in} \PY{n}{k\PYZus{}nearest\PYZus{}indexes}\PY{p}{]}\PY{p}{)}\PY{p}{)}
\end{Verbatim}


\begin{Verbatim}[commandchars=\\\{\}]
{\color{outcolor}Out[{\color{outcolor}22}]:} [(1385,
           0.2614510541240034,
           "He was the lawyer for Ted Collins' old Boston Yankees in the National Football League ."),
          (1386,
           0.6913933000758161,
           'All was quiet in the office of the Yankees and the local National Leaguers yesterday .'),
          (1369,
           0.6913933000758161,
           'His goal was to obtain a National League team for this city .'),
          (1457,
           0.7388835160664532,
           'A formula to supply players for the new Minneapolis Vikings and the problem of increasing the 1961 schedule to fourteen games will be discussed by National Football League owners at a meeting at the Hotel Warwick today .'),
          (5636,
           0.7388835160664532,
           'When he came to Baltimore , he was leaving a team which was supposed to win the National League pennant , and he was joining what seemed to be a second division American League club .'),
          (1367,
           0.7388835160664532,
           'The 53-year-old Shea , a prominent corporation lawyer with a sports background , is generally recognized as the man most responsible for the imminent return of a National League club to New York .'),
          (1340,
           0.7446230407723753,
           'Skorich was considered the logical choice after the club gave Norm Van Brocklin permission to seek the head coaching job with the Minnesota Vikings , the newest National Football League entry .'),
          (1372,
           0.7538170180413345,
           'When he was unable to bring about immediate expansion , he sought to convince another National League club to move here .'),
          (19177,
           0.7592282938284616,
           'Nearly any lad with a modicum of skill might find a payday awaiting him in the Three I League , or the Pony League , or the Coastal Plains League , or the fast Eastern League , if not indeed in one of the hundreds of city leagues that abounded everywhere .'),
          (48721, 0.7642977396044841, "I'm no lawyer .")]
\end{Verbatim}
            
    \hypertarget{search-on-histogram-of-colors}{%
\subsection{ Search on Histogram of
Colors}\label{search-on-histogram-of-colors}}

Now we will use the same function, \texttt{k\_neighbours}, to search
over image vector spaces.

    \begin{Verbatim}[commandchars=\\\{\}]
{\color{incolor}In [{\color{incolor}23}]:} \PY{c+c1}{\PYZsh{}Images database}
         \PY{n}{images} \PY{o}{=} \PY{p}{[}\PY{n}{image1}\PY{p}{,} \PY{n}{image2}\PY{p}{,} \PY{n}{image3}\PY{p}{,} \PY{n}{image4}\PY{p}{,} \PY{n}{image5}\PY{p}{,} \PY{n}{image6}\PY{p}{,} \PY{n}{image7}\PY{p}{,} \PY{n}{image8}\PY{p}{,} \PY{n}{image9}\PY{p}{,} \PY{n}{image10}\PY{p}{,} \PY{n}{image15}\PY{p}{]}
         \PY{n}{bins}\PY{o}{=}\PY{p}{(}\PY{l+m+mi}{4}\PY{p}{,}\PY{l+m+mi}{4}\PY{p}{,}\PY{l+m+mi}{4}\PY{p}{)}
\end{Verbatim}


    \begin{Verbatim}[commandchars=\\\{\}]
{\color{incolor}In [{\color{incolor}24}]:} \PY{k+kn}{from} \PY{n+nn}{skimage} \PY{k}{import} \PY{n}{img\PYZus{}as\PYZus{}ubyte}
         \PY{k+kn}{from} \PY{n+nn}{sklearn}\PY{n+nn}{.}\PY{n+nn}{preprocessing} \PY{k}{import} \PY{n}{normalize}
         \PY{k+kn}{from} \PY{n+nn}{skimage} \PY{k}{import} \PY{n}{color}
         
         \PY{n}{feats} \PY{o}{=} \PY{p}{[}\PY{p}{]}
         \PY{k}{for} \PY{n}{img} \PY{o+ow}{in} \PY{n}{images}\PY{p}{:}
             \PY{n}{img} \PY{o}{=} \PY{n}{imread}\PY{p}{(}\PY{l+s+s2}{\PYZdq{}}\PY{l+s+s2}{data/}\PY{l+s+s2}{\PYZdq{}} \PY{o}{+} \PY{n}{img}\PY{p}{)}
             
             \PY{c+c1}{\PYZsh{} resize image}
             \PY{n}{img} \PY{o}{=} \PY{n}{center\PYZus{}crop\PYZus{}image}\PY{p}{(}\PY{n}{img}\PY{p}{,} \PY{n}{size}\PY{o}{=}\PY{l+m+mi}{224}\PY{p}{)}
             
             \PY{c+c1}{\PYZsh{} Change image color space from RGB to HSV. }
             \PY{c+c1}{\PYZsh{} HSV color space was designed to more closely align with the way human vision perceive color\PYZhy{}making attributes}
             \PY{n}{img} \PY{o}{=} \PY{n}{color}\PY{o}{.}\PY{n}{rgb2hsv}\PY{p}{(}\PY{n}{img}\PY{p}{)}
             
             \PY{c+c1}{\PYZsh{} convert image pixels to [0, 255] range, and to uint8 type}
             \PY{n}{img\PYZus{}int} \PY{o}{=} \PY{n}{img\PYZus{}as\PYZus{}ubyte}\PY{p}{(}\PY{n}{img}\PY{p}{)}
         
             \PY{c+c1}{\PYZsh{} Extract HoC features}
             \PY{n}{hist}\PY{p}{,} \PY{n}{bin\PYZus{}edges} \PY{o}{=} \PY{n}{hoc}\PY{p}{(}\PY{n}{img\PYZus{}int}\PY{p}{,} \PY{n}{bins}\PY{o}{=}\PY{n}{bins}\PY{p}{)}
             
             \PY{c+c1}{\PYZsh{} Normalize features}
             \PY{c+c1}{\PYZsh{} We add 1 dimension to comply with scikit\PYZhy{}learn API}
             \PY{n}{hist} \PY{o}{=} \PY{n}{np}\PY{o}{.}\PY{n}{squeeze}\PY{p}{(}\PY{n}{normalize}\PY{p}{(}\PY{n}{hist}\PY{o}{.}\PY{n}{reshape}\PY{p}{(}\PY{l+m+mi}{1}\PY{p}{,} \PY{o}{\PYZhy{}}\PY{l+m+mi}{1}\PY{p}{)}\PY{p}{,} \PY{n}{norm}\PY{o}{=}\PY{l+s+s2}{\PYZdq{}}\PY{l+s+s2}{l2}\PY{l+s+s2}{\PYZdq{}}\PY{p}{)}\PY{p}{)}
             
             \PY{n}{feats}\PY{o}{.}\PY{n}{append}\PY{p}{(}\PY{n}{hist}\PY{p}{)}
             
         \PY{c+c1}{\PYZsh{} Creating a feature matrix for all images}
         \PY{n}{feats} \PY{o}{=} \PY{n}{np}\PY{o}{.}\PY{n}{array}\PY{p}{(}\PY{n}{feats}\PY{p}{)}
         
         \PY{n+nb}{print}\PY{p}{(}\PY{l+s+s2}{\PYZdq{}}\PY{l+s+s2}{Shape of feature matrix: }\PY{l+s+si}{\PYZob{}\PYZcb{}}\PY{l+s+s2}{\PYZdq{}}\PY{o}{.}\PY{n}{format}\PY{p}{(}\PY{n}{feats}\PY{o}{.}\PY{n}{shape}\PY{p}{)}\PY{p}{)}
\end{Verbatim}


    \begin{Verbatim}[commandchars=\\\{\}]
Shape of feature matrix: (11, 64)

    \end{Verbatim}

    \begin{Verbatim}[commandchars=\\\{\}]
{\color{incolor}In [{\color{incolor}25}]:} \PY{c+c1}{\PYZsh{} Prepare image query}
         \PY{c+c1}{\PYZsh{} try the following: image 11, image12, image13}
         \PY{n}{image\PYZus{}q} \PY{o}{=} \PY{n}{image13}
         
         \PY{n}{img\PYZus{}q} \PY{o}{=} \PY{n}{imread}\PY{p}{(}\PY{l+s+s2}{\PYZdq{}}\PY{l+s+s2}{data/}\PY{l+s+s2}{\PYZdq{}} \PY{o}{+} \PY{n}{image\PYZus{}q}\PY{p}{)}
         \PY{n}{img\PYZus{}q\PYZus{}hsv} \PY{o}{=} \PY{n}{color}\PY{o}{.}\PY{n}{rgb2hsv}\PY{p}{(}\PY{n}{img\PYZus{}q}\PY{p}{)}
         \PY{n}{img\PYZus{}q\PYZus{}hsv} \PY{o}{=} \PY{n}{center\PYZus{}crop\PYZus{}image}\PY{p}{(}\PY{n}{img\PYZus{}q\PYZus{}hsv}\PY{p}{,} \PY{n}{size}\PY{o}{=}\PY{l+m+mi}{224}\PY{p}{)}
         \PY{n}{img\PYZus{}int} \PY{o}{=} \PY{n}{img\PYZus{}as\PYZus{}ubyte}\PY{p}{(}\PY{n}{img\PYZus{}q\PYZus{}hsv}\PY{p}{)}
         \PY{n}{hist}\PY{p}{,} \PY{n}{bin\PYZus{}edges} \PY{o}{=} \PY{n}{hoc}\PY{p}{(}\PY{n}{img\PYZus{}int}\PY{p}{,} \PY{n}{bins}\PY{o}{=}\PY{n}{bins}\PY{p}{)}
         
         \PY{n}{image\PYZus{}q\PYZus{}feat} \PY{o}{=} \PY{n}{np}\PY{o}{.}\PY{n}{squeeze}\PY{p}{(}\PY{n}{normalize}\PY{p}{(}\PY{n}{hist}\PY{o}{.}\PY{n}{reshape}\PY{p}{(}\PY{l+m+mi}{1}\PY{p}{,} \PY{o}{\PYZhy{}}\PY{l+m+mi}{1}\PY{p}{)}\PY{p}{,} \PY{n}{norm}\PY{o}{=}\PY{l+s+s2}{\PYZdq{}}\PY{l+s+s2}{l2}\PY{l+s+s2}{\PYZdq{}}\PY{p}{)}\PY{p}{)}
         \PY{n+nb}{print}\PY{p}{(}\PY{l+s+s2}{\PYZdq{}}\PY{l+s+s2}{Image query feature dimension: }\PY{l+s+si}{\PYZob{}\PYZcb{}}\PY{l+s+s2}{\PYZdq{}}\PY{o}{.}\PY{n}{format}\PY{p}{(}\PY{n}{image\PYZus{}q\PYZus{}feat}\PY{o}{.}\PY{n}{shape}\PY{p}{)}\PY{p}{)}
\end{Verbatim}


    \begin{Verbatim}[commandchars=\\\{\}]
Image query feature dimension: (64,)

    \end{Verbatim}

    \begin{Verbatim}[commandchars=\\\{\}]
{\color{incolor}In [{\color{incolor}26}]:} \PY{c+c1}{\PYZsh{} Use the implemented function to find the K nearest neighbours on the HoC image vector space.}
         \PY{n}{k\PYZus{}nearest\PYZus{}indexes}\PY{p}{,} \PY{n}{k\PYZus{}nearest\PYZus{}dists} \PY{o}{=} \PY{n}{k\PYZus{}neighbours}\PY{p}{(}\PY{n}{q}\PY{o}{=}\PY{n}{image\PYZus{}q\PYZus{}feat}\PY{o}{.}\PY{n}{reshape}\PY{p}{(}\PY{l+m+mi}{1}\PY{p}{,}\PY{o}{\PYZhy{}}\PY{l+m+mi}{1}\PY{p}{)}\PY{p}{,} \PY{n}{X}\PY{o}{=}\PY{n}{feats}\PY{p}{,} \PY{n}{metric}\PY{o}{=}\PY{l+s+s2}{\PYZdq{}}\PY{l+s+s2}{euclidean}\PY{l+s+s2}{\PYZdq{}}\PY{p}{,} \PY{n}{k}\PY{o}{=}\PY{l+m+mi}{10}\PY{p}{)}
\end{Verbatim}


    \begin{Verbatim}[commandchars=\\\{\}]
{\color{incolor}In [{\color{incolor}27}]:} \PY{c+c1}{\PYZsh{} Inspecting the top\PYZhy{}k results \PYZhy{} list of tuples (image index, distance to query image, image name)}
         \PY{n+nb}{list}\PY{p}{(}\PY{n+nb}{zip}\PY{p}{(}\PY{n}{k\PYZus{}nearest\PYZus{}indexes}\PY{p}{,} \PY{n}{k\PYZus{}nearest\PYZus{}dists}\PY{p}{,} \PY{p}{[}\PY{n}{images}\PY{p}{[}\PY{n}{i}\PY{p}{]} \PY{k}{for} \PY{n}{i} \PY{o+ow}{in} \PY{n}{k\PYZus{}nearest\PYZus{}indexes}\PY{p}{]}\PY{p}{)}\PY{p}{)}
\end{Verbatim}


\begin{Verbatim}[commandchars=\\\{\}]
{\color{outcolor}Out[{\color{outcolor}27}]:} [(10, 0.845573192138205, 'red2.jpg'),
          (6, 0.9863331777515734, 'strawberries.jpg'),
          (8, 1.0448437741073637, 'bird2.jpg'),
          (9, 1.0707964644610366, 'street2.jpg'),
          (4, 1.1160208091438566, 'cup.jpg'),
          (1, 1.19646013334974, 'lena.jpg'),
          (3, 1.2149713547356729, 'bird.jpg'),
          (7, 1.2277823976745124, 'ferrari.jpg'),
          (0, 1.2329417477994196, 'cars.jpg'),
          (2, 1.2724448802449535, 'street.jpg')]
\end{Verbatim}
            
    \hypertarget{inspect-results-of-image-search}{%
\subsubsection{Inspect Results of Image
Search}\label{inspect-results-of-image-search}}

    \begin{Verbatim}[commandchars=\\\{\}]
{\color{incolor}In [{\color{incolor}28}]:} \PY{n+nb}{print}\PY{p}{(}\PY{l+s+s2}{\PYZdq{}}\PY{l+s+s2}{Query image: }\PY{l+s+si}{\PYZob{}\PYZcb{}}\PY{l+s+s2}{\PYZdq{}}\PY{o}{.}\PY{n}{format}\PY{p}{(}\PY{n}{image\PYZus{}q}\PY{p}{)}\PY{p}{)}
         \PY{n}{plt}\PY{o}{.}\PY{n}{imshow}\PY{p}{(}\PY{n}{img\PYZus{}q}\PY{p}{)}
         \PY{n}{plt}\PY{o}{.}\PY{n}{axis}\PY{p}{(}\PY{l+s+s1}{\PYZsq{}}\PY{l+s+s1}{off}\PY{l+s+s1}{\PYZsq{}}\PY{p}{)}
         \PY{n}{plt}\PY{o}{.}\PY{n}{show}\PY{p}{(}\PY{p}{)}
         
         \PY{k}{for} \PY{n}{i}\PY{p}{,} \PY{p}{(}\PY{n}{img\PYZus{}idx}\PY{p}{,} \PY{n}{img\PYZus{}dist}\PY{p}{)} \PY{o+ow}{in} \PY{n+nb}{enumerate}\PY{p}{(}\PY{n+nb}{zip}\PY{p}{(}\PY{n}{k\PYZus{}nearest\PYZus{}indexes}\PY{p}{,}\PY{n}{k\PYZus{}nearest\PYZus{}dists}\PY{p}{)}\PY{p}{)}\PY{p}{:}
             \PY{n}{image\PYZus{}fname} \PY{o}{=} \PY{n}{images}\PY{p}{[}\PY{n}{img\PYZus{}idx}\PY{p}{]}
             \PY{n}{img} \PY{o}{=} \PY{n}{imread}\PY{p}{(}\PY{l+s+s2}{\PYZdq{}}\PY{l+s+s2}{data/}\PY{l+s+s2}{\PYZdq{}} \PY{o}{+} \PY{n}{image\PYZus{}fname}\PY{p}{)}
             \PY{n}{img} \PY{o}{=} \PY{n}{center\PYZus{}crop\PYZus{}image}\PY{p}{(}\PY{n}{img}\PY{p}{,} \PY{n}{size}\PY{o}{=}\PY{l+m+mi}{224}\PY{p}{)}
             \PY{n+nb}{print}\PY{p}{(}\PY{l+s+s2}{\PYZdq{}}\PY{l+s+s2}{Image \PYZsh{}}\PY{l+s+si}{\PYZob{}\PYZcb{}}\PY{l+s+s2}{ \PYZhy{} distance: }\PY{l+s+si}{\PYZob{}\PYZcb{}}\PY{l+s+s2}{\PYZdq{}}\PY{o}{.}\PY{n}{format}\PY{p}{(}\PY{n}{i}\PY{p}{,} \PY{n}{img\PYZus{}dist}\PY{p}{)}\PY{p}{)}
             \PY{n}{plt}\PY{o}{.}\PY{n}{imshow}\PY{p}{(}\PY{n}{img}\PY{p}{)}
             \PY{n}{plt}\PY{o}{.}\PY{n}{axis}\PY{p}{(}\PY{l+s+s1}{\PYZsq{}}\PY{l+s+s1}{off}\PY{l+s+s1}{\PYZsq{}}\PY{p}{)}
             \PY{n}{plt}\PY{o}{.}\PY{n}{show}\PY{p}{(}\PY{p}{)}
             
\end{Verbatim}


    \begin{Verbatim}[commandchars=\\\{\}]
Query image: flowers.jpg

    \end{Verbatim}

    \begin{center}
    \adjustimage{max size={0.9\linewidth}{0.9\paperheight}}{output_50_1.png}
    \end{center}
    { \hspace*{\fill} \\}
    
    \begin{Verbatim}[commandchars=\\\{\}]
Image \#0 - distance: 0.845573192138205

    \end{Verbatim}

    \begin{center}
    \adjustimage{max size={0.9\linewidth}{0.9\paperheight}}{output_50_3.png}
    \end{center}
    { \hspace*{\fill} \\}
    
    \begin{Verbatim}[commandchars=\\\{\}]
Image \#1 - distance: 0.9863331777515734

    \end{Verbatim}

    \begin{center}
    \adjustimage{max size={0.9\linewidth}{0.9\paperheight}}{output_50_5.png}
    \end{center}
    { \hspace*{\fill} \\}
    
    \begin{Verbatim}[commandchars=\\\{\}]
Image \#2 - distance: 1.0448437741073637

    \end{Verbatim}

    \begin{center}
    \adjustimage{max size={0.9\linewidth}{0.9\paperheight}}{output_50_7.png}
    \end{center}
    { \hspace*{\fill} \\}
    
    \begin{Verbatim}[commandchars=\\\{\}]
Image \#3 - distance: 1.0707964644610366

    \end{Verbatim}

    \begin{center}
    \adjustimage{max size={0.9\linewidth}{0.9\paperheight}}{output_50_9.png}
    \end{center}
    { \hspace*{\fill} \\}
    
    \begin{Verbatim}[commandchars=\\\{\}]
Image \#4 - distance: 1.1160208091438566

    \end{Verbatim}

    \begin{center}
    \adjustimage{max size={0.9\linewidth}{0.9\paperheight}}{output_50_11.png}
    \end{center}
    { \hspace*{\fill} \\}
    
    \begin{Verbatim}[commandchars=\\\{\}]
Image \#5 - distance: 1.19646013334974

    \end{Verbatim}

    \begin{center}
    \adjustimage{max size={0.9\linewidth}{0.9\paperheight}}{output_50_13.png}
    \end{center}
    { \hspace*{\fill} \\}
    
    \begin{Verbatim}[commandchars=\\\{\}]
Image \#6 - distance: 1.2149713547356729

    \end{Verbatim}

    \begin{center}
    \adjustimage{max size={0.9\linewidth}{0.9\paperheight}}{output_50_15.png}
    \end{center}
    { \hspace*{\fill} \\}
    
    \begin{Verbatim}[commandchars=\\\{\}]
Image \#7 - distance: 1.2277823976745124

    \end{Verbatim}

    \begin{center}
    \adjustimage{max size={0.9\linewidth}{0.9\paperheight}}{output_50_17.png}
    \end{center}
    { \hspace*{\fill} \\}
    
    \begin{Verbatim}[commandchars=\\\{\}]
Image \#8 - distance: 1.2329417477994196

    \end{Verbatim}

    \begin{center}
    \adjustimage{max size={0.9\linewidth}{0.9\paperheight}}{output_50_19.png}
    \end{center}
    { \hspace*{\fill} \\}
    
    \begin{Verbatim}[commandchars=\\\{\}]
Image \#9 - distance: 1.2724448802449535

    \end{Verbatim}

    \begin{center}
    \adjustimage{max size={0.9\linewidth}{0.9\paperheight}}{output_50_21.png}
    \end{center}
    { \hspace*{\fill} \\}
    
    \hypertarget{search-on-hog-vector-space}{%
\subsection{ Search on HoG vector
space}\label{search-on-hog-vector-space}}

    \begin{Verbatim}[commandchars=\\\{\}]
{\color{incolor}In [{\color{incolor}29}]:} \PY{c+c1}{\PYZsh{}Images database}
         \PY{n}{images} \PY{o}{=} \PY{p}{[}\PY{n}{image1}\PY{p}{,} \PY{n}{image2}\PY{p}{,} \PY{n}{image3}\PY{p}{,} \PY{n}{image4}\PY{p}{,} \PY{n}{image5}\PY{p}{,} \PY{n}{image6}\PY{p}{,} \PY{n}{image7}\PY{p}{,} \PY{n}{image8}\PY{p}{,} \PY{n}{image9}\PY{p}{,} \PY{n}{image10}\PY{p}{,} \PY{n}{image15}\PY{p}{]}
         
         \PY{n}{pixels\PYZus{}per\PYZus{}cell}\PY{o}{=}\PY{p}{(}\PY{l+m+mi}{32}\PY{p}{,}\PY{l+m+mi}{32}\PY{p}{)}
         \PY{n}{orientations}\PY{o}{=}\PY{l+m+mi}{8}
\end{Verbatim}


    \begin{Verbatim}[commandchars=\\\{\}]
{\color{incolor}In [{\color{incolor}30}]:} \PY{k+kn}{from} \PY{n+nn}{skimage} \PY{k}{import} \PY{n}{img\PYZus{}as\PYZus{}ubyte}
         \PY{k+kn}{from} \PY{n+nn}{sklearn}\PY{n+nn}{.}\PY{n+nn}{preprocessing} \PY{k}{import} \PY{n}{normalize}
         \PY{k+kn}{from} \PY{n+nn}{skimage} \PY{k}{import} \PY{n}{color}
         
         \PY{n}{feats} \PY{o}{=} \PY{p}{[}\PY{p}{]}
         \PY{k}{for} \PY{n}{img} \PY{o+ow}{in} \PY{n}{images}\PY{p}{:}
             \PY{n}{img} \PY{o}{=} \PY{n}{imread}\PY{p}{(}\PY{l+s+s2}{\PYZdq{}}\PY{l+s+s2}{data/}\PY{l+s+s2}{\PYZdq{}} \PY{o}{+} \PY{n}{img}\PY{p}{)}
             
             \PY{c+c1}{\PYZsh{} resize image}
             \PY{n}{img} \PY{o}{=} \PY{n}{center\PYZus{}crop\PYZus{}image}\PY{p}{(}\PY{n}{img}\PY{p}{,} \PY{n}{size}\PY{o}{=}\PY{l+m+mi}{224}\PY{p}{)}
             
             \PY{c+c1}{\PYZsh{} Convert to grayscale}
             \PY{n}{img} \PY{o}{=} \PY{n}{rgb2gray}\PY{p}{(}\PY{n}{img}\PY{p}{)}
             
             \PY{c+c1}{\PYZsh{} Extract HoG features}
             \PY{n}{hist} \PY{o}{=} \PY{n}{hog}\PY{p}{(}\PY{n}{img}\PY{p}{,} \PY{n}{orientations}\PY{o}{=}\PY{n}{orientations}\PY{p}{,} \PY{n}{pixels\PYZus{}per\PYZus{}cell}\PY{o}{=}\PY{n}{pixels\PYZus{}per\PYZus{}cell}\PY{p}{)}
             
             \PY{c+c1}{\PYZsh{} Normalize features}
             \PY{c+c1}{\PYZsh{} We add 1 dimension to comply with scikit\PYZhy{}learn API}
             \PY{n}{hist} \PY{o}{=} \PY{n}{np}\PY{o}{.}\PY{n}{squeeze}\PY{p}{(}\PY{n}{normalize}\PY{p}{(}\PY{n}{hist}\PY{o}{.}\PY{n}{reshape}\PY{p}{(}\PY{l+m+mi}{1}\PY{p}{,} \PY{o}{\PYZhy{}}\PY{l+m+mi}{1}\PY{p}{)}\PY{p}{,} \PY{n}{norm}\PY{o}{=}\PY{l+s+s2}{\PYZdq{}}\PY{l+s+s2}{l2}\PY{l+s+s2}{\PYZdq{}}\PY{p}{)}\PY{p}{)}
             
             \PY{n}{feats}\PY{o}{.}\PY{n}{append}\PY{p}{(}\PY{n}{hist}\PY{p}{)}
             
         \PY{c+c1}{\PYZsh{} Creating a feature matrix for all images}
         \PY{n}{feats} \PY{o}{=} \PY{n}{np}\PY{o}{.}\PY{n}{array}\PY{p}{(}\PY{n}{feats}\PY{p}{)}
         
         \PY{n+nb}{print}\PY{p}{(}\PY{l+s+s2}{\PYZdq{}}\PY{l+s+s2}{Shape of feature matrix: }\PY{l+s+si}{\PYZob{}\PYZcb{}}\PY{l+s+s2}{\PYZdq{}}\PY{o}{.}\PY{n}{format}\PY{p}{(}\PY{n}{feats}\PY{o}{.}\PY{n}{shape}\PY{p}{)}\PY{p}{)}
\end{Verbatim}


    \begin{Verbatim}[commandchars=\\\{\}]
Shape of feature matrix: (11, 1800)

    \end{Verbatim}

    \begin{Verbatim}[commandchars=\\\{\}]
{\color{incolor}In [{\color{incolor}31}]:} \PY{c+c1}{\PYZsh{} Prepare image query}
         \PY{c+c1}{\PYZsh{} try the following: image 11, image12, image13}
         \PY{n}{image\PYZus{}q} \PY{o}{=} \PY{n}{image11}
         
         \PY{n}{img\PYZus{}q} \PY{o}{=} \PY{n}{imread}\PY{p}{(}\PY{l+s+s2}{\PYZdq{}}\PY{l+s+s2}{data/}\PY{l+s+s2}{\PYZdq{}} \PY{o}{+} \PY{n}{image\PYZus{}q}\PY{p}{)}
         \PY{n}{img\PYZus{}q} \PY{o}{=} \PY{n}{center\PYZus{}crop\PYZus{}image}\PY{p}{(}\PY{n}{img\PYZus{}q}\PY{p}{,} \PY{n}{size}\PY{o}{=}\PY{l+m+mi}{224}\PY{p}{)}
         \PY{n}{img\PYZus{}q} \PY{o}{=} \PY{n}{rgb2gray}\PY{p}{(}\PY{n}{img\PYZus{}q}\PY{p}{)}
         
         \PY{n}{hist} \PY{o}{=} \PY{n}{hog}\PY{p}{(}\PY{n}{img\PYZus{}q}\PY{p}{,} \PY{n}{orientations}\PY{o}{=}\PY{n}{orientations}\PY{p}{,} \PY{n}{pixels\PYZus{}per\PYZus{}cell}\PY{o}{=}\PY{n}{pixels\PYZus{}per\PYZus{}cell}\PY{p}{)}
         \PY{n}{image\PYZus{}q\PYZus{}feat} \PY{o}{=} \PY{n}{np}\PY{o}{.}\PY{n}{squeeze}\PY{p}{(}\PY{n}{normalize}\PY{p}{(}\PY{n}{hist}\PY{o}{.}\PY{n}{reshape}\PY{p}{(}\PY{l+m+mi}{1}\PY{p}{,} \PY{o}{\PYZhy{}}\PY{l+m+mi}{1}\PY{p}{)}\PY{p}{,} \PY{n}{norm}\PY{o}{=}\PY{l+s+s2}{\PYZdq{}}\PY{l+s+s2}{l2}\PY{l+s+s2}{\PYZdq{}}\PY{p}{)}\PY{p}{)}
         
         \PY{n+nb}{print}\PY{p}{(}\PY{l+s+s2}{\PYZdq{}}\PY{l+s+s2}{Image query feature dimension: }\PY{l+s+si}{\PYZob{}\PYZcb{}}\PY{l+s+s2}{\PYZdq{}}\PY{o}{.}\PY{n}{format}\PY{p}{(}\PY{n}{image\PYZus{}q\PYZus{}feat}\PY{o}{.}\PY{n}{shape}\PY{p}{)}\PY{p}{)}
\end{Verbatim}


    \begin{Verbatim}[commandchars=\\\{\}]
Image query feature dimension: (1800,)

    \end{Verbatim}

    \begin{Verbatim}[commandchars=\\\{\}]
{\color{incolor}In [{\color{incolor}32}]:} \PY{c+c1}{\PYZsh{} Use the implemented function to find the K nearest neighbours on the HoG image vector space.}
         \PY{n}{k\PYZus{}nearest\PYZus{}indexes}\PY{p}{,} \PY{n}{k\PYZus{}nearest\PYZus{}dists} \PY{o}{=} \PY{n}{k\PYZus{}neighbours}\PY{p}{(}\PY{n}{q}\PY{o}{=}\PY{n}{image\PYZus{}q\PYZus{}feat}\PY{o}{.}\PY{n}{reshape}\PY{p}{(}\PY{l+m+mi}{1}\PY{p}{,}\PY{o}{\PYZhy{}}\PY{l+m+mi}{1}\PY{p}{)}\PY{p}{,} \PY{n}{X}\PY{o}{=}\PY{n}{feats}\PY{p}{,} \PY{n}{metric}\PY{o}{=}\PY{l+s+s2}{\PYZdq{}}\PY{l+s+s2}{euclidean}\PY{l+s+s2}{\PYZdq{}}\PY{p}{,} \PY{n}{k}\PY{o}{=}\PY{l+m+mi}{10}\PY{p}{)}
\end{Verbatim}


    \begin{Verbatim}[commandchars=\\\{\}]
{\color{incolor}In [{\color{incolor}33}]:} \PY{c+c1}{\PYZsh{} Inspecting the top\PYZhy{}k results \PYZhy{} list of tuples (image index, distance to query image, image name)}
         \PY{n+nb}{list}\PY{p}{(}\PY{n+nb}{zip}\PY{p}{(}\PY{n}{k\PYZus{}nearest\PYZus{}indexes}\PY{p}{,} \PY{n}{k\PYZus{}nearest\PYZus{}dists}\PY{p}{,} \PY{p}{[}\PY{n}{images}\PY{p}{[}\PY{n}{i}\PY{p}{]} \PY{k}{for} \PY{n}{i} \PY{o+ow}{in} \PY{n}{k\PYZus{}nearest\PYZus{}indexes}\PY{p}{]}\PY{p}{)}\PY{p}{)}
\end{Verbatim}


\begin{Verbatim}[commandchars=\\\{\}]
{\color{outcolor}Out[{\color{outcolor}33}]:} [(0, 0.7463698434976643, 'cars.jpg'),
          (9, 0.7735160568069858, 'street2.jpg'),
          (6, 0.7914558493872327, 'strawberries.jpg'),
          (2, 0.8637713919432636, 'street.jpg'),
          (7, 0.8880615143588726, 'ferrari.jpg'),
          (8, 0.8979937220959415, 'bird2.jpg'),
          (1, 0.9084086141214707, 'lena.jpg'),
          (10, 0.9166115885416387, 'red2.jpg'),
          (4, 0.9417340667750381, 'cup.jpg'),
          (5, 0.9999999999999998, 'red.jpg')]
\end{Verbatim}
            
    \hypertarget{inspect-results-of-image-search}{%
\subsection{Inspect Results of Image
Search}\label{inspect-results-of-image-search}}

    \begin{Verbatim}[commandchars=\\\{\}]
{\color{incolor}In [{\color{incolor}34}]:} \PY{n+nb}{print}\PY{p}{(}\PY{l+s+s2}{\PYZdq{}}\PY{l+s+s2}{Query image: }\PY{l+s+si}{\PYZob{}\PYZcb{}}\PY{l+s+s2}{\PYZdq{}}\PY{o}{.}\PY{n}{format}\PY{p}{(}\PY{n}{image\PYZus{}q}\PY{p}{)}\PY{p}{)}
         \PY{n}{plt}\PY{o}{.}\PY{n}{imshow}\PY{p}{(}\PY{n}{img\PYZus{}q}\PY{p}{,} \PY{n}{cmap}\PY{o}{=}\PY{n}{plt}\PY{o}{.}\PY{n}{cm}\PY{o}{.}\PY{n}{gray}\PY{p}{)}
         \PY{n}{plt}\PY{o}{.}\PY{n}{axis}\PY{p}{(}\PY{l+s+s1}{\PYZsq{}}\PY{l+s+s1}{off}\PY{l+s+s1}{\PYZsq{}}\PY{p}{)}
         \PY{n}{plt}\PY{o}{.}\PY{n}{show}\PY{p}{(}\PY{p}{)}
         
         \PY{k}{for} \PY{n}{i}\PY{p}{,} \PY{p}{(}\PY{n}{img\PYZus{}idx}\PY{p}{,} \PY{n}{img\PYZus{}dist}\PY{p}{)} \PY{o+ow}{in} \PY{n+nb}{enumerate}\PY{p}{(}\PY{n+nb}{zip}\PY{p}{(}\PY{n}{k\PYZus{}nearest\PYZus{}indexes}\PY{p}{,}\PY{n}{k\PYZus{}nearest\PYZus{}dists}\PY{p}{)}\PY{p}{)}\PY{p}{:}
             \PY{n}{image\PYZus{}fname} \PY{o}{=} \PY{n}{images}\PY{p}{[}\PY{n}{img\PYZus{}idx}\PY{p}{]}
             \PY{n}{img} \PY{o}{=} \PY{n}{imread}\PY{p}{(}\PY{l+s+s2}{\PYZdq{}}\PY{l+s+s2}{data/}\PY{l+s+s2}{\PYZdq{}} \PY{o}{+} \PY{n}{image\PYZus{}fname}\PY{p}{)}
             \PY{n}{img} \PY{o}{=} \PY{n}{center\PYZus{}crop\PYZus{}image}\PY{p}{(}\PY{n}{img}\PY{p}{,} \PY{n}{size}\PY{o}{=}\PY{l+m+mi}{224}\PY{p}{)}
             \PY{n}{img} \PY{o}{=} \PY{n}{rgb2gray}\PY{p}{(}\PY{n}{img}\PY{p}{)}
             \PY{n+nb}{print}\PY{p}{(}\PY{l+s+s2}{\PYZdq{}}\PY{l+s+s2}{Image \PYZsh{}}\PY{l+s+si}{\PYZob{}\PYZcb{}}\PY{l+s+s2}{ \PYZhy{} distance: }\PY{l+s+si}{\PYZob{}\PYZcb{}}\PY{l+s+s2}{\PYZdq{}}\PY{o}{.}\PY{n}{format}\PY{p}{(}\PY{n}{i}\PY{p}{,} \PY{n}{img\PYZus{}dist}\PY{p}{)}\PY{p}{)}
             \PY{n}{plt}\PY{o}{.}\PY{n}{imshow}\PY{p}{(}\PY{n}{img}\PY{p}{,} \PY{n}{cmap}\PY{o}{=}\PY{n}{plt}\PY{o}{.}\PY{n}{cm}\PY{o}{.}\PY{n}{gray}\PY{p}{)}
             \PY{n}{plt}\PY{o}{.}\PY{n}{axis}\PY{p}{(}\PY{l+s+s1}{\PYZsq{}}\PY{l+s+s1}{off}\PY{l+s+s1}{\PYZsq{}}\PY{p}{)}
             \PY{n}{plt}\PY{o}{.}\PY{n}{show}\PY{p}{(}\PY{p}{)}
             
\end{Verbatim}


    \begin{Verbatim}[commandchars=\\\{\}]
Query image: car2.jpg

    \end{Verbatim}

    \begin{center}
    \adjustimage{max size={0.9\linewidth}{0.9\paperheight}}{output_58_1.png}
    \end{center}
    { \hspace*{\fill} \\}
    
    \begin{Verbatim}[commandchars=\\\{\}]
Image \#0 - distance: 0.7463698434976643

    \end{Verbatim}

    \begin{center}
    \adjustimage{max size={0.9\linewidth}{0.9\paperheight}}{output_58_3.png}
    \end{center}
    { \hspace*{\fill} \\}
    
    \begin{Verbatim}[commandchars=\\\{\}]
Image \#1 - distance: 0.7735160568069858

    \end{Verbatim}

    \begin{center}
    \adjustimage{max size={0.9\linewidth}{0.9\paperheight}}{output_58_5.png}
    \end{center}
    { \hspace*{\fill} \\}
    
    \begin{Verbatim}[commandchars=\\\{\}]
Image \#2 - distance: 0.7914558493872327

    \end{Verbatim}

    \begin{center}
    \adjustimage{max size={0.9\linewidth}{0.9\paperheight}}{output_58_7.png}
    \end{center}
    { \hspace*{\fill} \\}
    
    \begin{Verbatim}[commandchars=\\\{\}]
Image \#3 - distance: 0.8637713919432636

    \end{Verbatim}

    \begin{center}
    \adjustimage{max size={0.9\linewidth}{0.9\paperheight}}{output_58_9.png}
    \end{center}
    { \hspace*{\fill} \\}
    
    \begin{Verbatim}[commandchars=\\\{\}]
Image \#4 - distance: 0.8880615143588726

    \end{Verbatim}

    \begin{center}
    \adjustimage{max size={0.9\linewidth}{0.9\paperheight}}{output_58_11.png}
    \end{center}
    { \hspace*{\fill} \\}
    
    \begin{Verbatim}[commandchars=\\\{\}]
Image \#5 - distance: 0.8979937220959415

    \end{Verbatim}

    \begin{center}
    \adjustimage{max size={0.9\linewidth}{0.9\paperheight}}{output_58_13.png}
    \end{center}
    { \hspace*{\fill} \\}
    
    \begin{Verbatim}[commandchars=\\\{\}]
Image \#6 - distance: 0.9084086141214707

    \end{Verbatim}

    \begin{center}
    \adjustimage{max size={0.9\linewidth}{0.9\paperheight}}{output_58_15.png}
    \end{center}
    { \hspace*{\fill} \\}
    
    \begin{Verbatim}[commandchars=\\\{\}]
Image \#7 - distance: 0.9166115885416387

    \end{Verbatim}

    \begin{center}
    \adjustimage{max size={0.9\linewidth}{0.9\paperheight}}{output_58_17.png}
    \end{center}
    { \hspace*{\fill} \\}
    
    \begin{Verbatim}[commandchars=\\\{\}]
Image \#8 - distance: 0.9417340667750381

    \end{Verbatim}

    \begin{center}
    \adjustimage{max size={0.9\linewidth}{0.9\paperheight}}{output_58_19.png}
    \end{center}
    { \hspace*{\fill} \\}
    
    \begin{Verbatim}[commandchars=\\\{\}]
Image \#9 - distance: 0.9999999999999998

    \end{Verbatim}

    \begin{center}
    \adjustimage{max size={0.9\linewidth}{0.9\paperheight}}{output_58_21.png}
    \end{center}
    { \hspace*{\fill} \\}
    
    \hypertarget{exercises}{%
\subsection{ Exercises}\label{exercises}}

    \begin{Verbatim}[commandchars=\\\{\}]
{\color{incolor}In [{\color{incolor}46}]:} \PY{c+c1}{\PYZsh{} Plot the color histogram of each color channel. Use 32 bins per color channel.}
\end{Verbatim}


    \begin{Verbatim}[commandchars=\\\{\}]
{\color{incolor}In [{\color{incolor}41}]:} \PY{c+c1}{\PYZsh{} Inspect the BoW vocabulary and try to understand which additional filters you could apply to remove uninformative words.}
\end{Verbatim}


    \begin{Verbatim}[commandchars=\\\{\}]
{\color{incolor}In [{\color{incolor}43}]:} \PY{c+c1}{\PYZsh{} Modify or remove the normalization steps and repeat the experiments. Compare the results with the normalized version and discuss what you observe.}
\end{Verbatim}


    \begin{Verbatim}[commandchars=\\\{\}]
{\color{incolor}In [{\color{incolor}44}]:} \PY{c+c1}{\PYZsh{} Modify the metric for each space and repeat the experiments. Compare the results with the correct metric and discuss what you observe.}
\end{Verbatim}


    \begin{Verbatim}[commandchars=\\\{\}]
{\color{incolor}In [{\color{incolor}1}]:} \PY{c+c1}{\PYZsh{} Implement a visual search space with the GIST feature.}
\end{Verbatim}



    % Add a bibliography block to the postdoc
    
    
    
    \end{document}
